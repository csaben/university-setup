% --------------------------------------------------------------
% This is all preamble stuff that you don't have to worry about.
% Head down to where it says "Start here"
% --------------------------------------------------------------
 
\documentclass[12pt]{article}
 
\usepackage[margin=1in]{geometry} 
\usepackage{amsmath,amsthm,amssymb}
\usepackage{actuarialsymbol}
 
\newcommand{\N}{\mathbb{N}}
\newcommand{\Z}{\mathbb{Z}}
 
\newenvironment{theorem}[2][Theorem]{\begin{trivlist}
\item[\hskip \labelsep {\bfseries #1}\hskip \labelsep {\bfseries #2.}]}{\end{trivlist}}
\newenvironment{lemma}[2][Lemma]{\begin{trivlist}
\item[\hskip \labelsep {\bfseries #1}\hskip \labelsep {\bfseries #2.}]}{\end{trivlist}}
\newenvironment{exercise}[2][Exercise]{\begin{trivlist}
\item[\hskip \labelsep {\bfseries #1}\hskip \labelsep {\bfseries #2.}]}{\end{trivlist}}
\newenvironment{problem}[2][Problem]{\begin{trivlist}
\item[\hskip \labelsep {\bfseries #1}\hskip \labelsep {\bfseries #2.}]}{\end{trivlist}}
\newenvironment{question}[2][Question]{\begin{trivlist}
\item[\hskip \labelsep {\bfseries #1}\hskip \labelsep {\bfseries #2.}]}{\end{trivlist}}
\newenvironment{corollary}[2][Corollary]{\begin{trivlist}
\item[\hskip \labelsep {\bfseries #1}\hskip \labelsep {\bfseries #2.}]}{\end{trivlist}}
 
\begin{document}
 
% --------------------------------------------------------------
%                         Start here
% --------------------------------------------------------------
 
\title{Theory of Interest}%replace X with the appropriate number
\author{Clark Saben\\ %replace with your name
}% 3/6/23 lecture} %if necessary, replace with your course title
% \title{}%replace X with the appropriate number
%sample envs
% \begin{theorem}[Theorem 1.1]
% \begin{lemma}[Lemma 1.1]
% \begin{exercise}[Exercise 1.1]
% \begin{problem}[Problem 1.1]
% \begin{question}[Question 1.1]
% \begin{corollary}[Corollary 1.1]
\maketitle

\begin{section}{3/8/23 Lecture}

%iitemize
Homework and Logistics
\hline 
\begin{itemize}
	\item 3:30pm talk  (student engagement)
	\item PS3: 2.2.7, 2.2.11, 2.2.13 (first two on last lecture notes, 3rd on today), 
		Due: 6pm on Tuesday 3/14 with an extra problem given on monday. In total,
		4 problems.
\end{itemize}
\hline

Remark:\\

In general, \\
$ S_{\actuarialangle{n}
$ a_{\actuarialangle{n}}:=$ is the present value of "    ".\\
Thus,\\
$v^{k}a_{\actuarialangle{n}}:= a_{\actuarialangle{n+k}} - a_{\actuarialangle{k}}$\\

Imagine a number line with 0 and $k$ and $k+1$	on it.\

Therefore,\\
$a_{\actuarialangle{n+k}} = a_{\actuarialangle{k}} + v^{k}a_{\actuarialangle{n}}$\\

As we saw in the last lecture,\\
$S_{\actuarialangle{n+k}} = S_{\actuarialangle{n}}\left(1+i\right)^{k} + S_{\actuarialangle{n}}$.\\
Here,\\
$v^{k}a_{\actuarialangle{n}}$ is the PV of a k period deferred annuity immediate.\\

\begin{question}[Example 2.2.13]
	a Jim can make an investment of 10,000 in 2 ways:\\
	\begin{enumerate}
		\item Deposits into an account yielding an annual interest rate of $i$.
		\item He can purchase an annuity immediate (payments occur at end of month) with
			24 level payments (the amounts don't change) annually, at an 
			annual rate of 10\%. These payments are then deposited into
			a fund that yields an annual effective rate of 5\%. 
	\end{enumerate}
	If both options produce the same accumulated value at the end of 24 years
	what is the value of $i$?
\end{question}

\boxed{Solution:}\\


Option A:\\
\hline

We will determine the respective accumuluated values. And equate them, to find $i$.\\

Accumulated value of the first option:\\
$S_{\actuarialangle{24}} = 10,000\left(1+i\right)^{24}$\\

Accumulated value of the second option:\\

Let $X$ be the level payment.\\
Then, \\
$10,000 = Xa_{\actuarialangle{24}.1}$\\
Therefore,\\
$X = \frac{10,000}{a_{\actuarialangle{24}.1}} = \frac{10000}{\frac{1-v^{24}}{i}}$ \\
$i=10$ in the actuarial angle\%\\
What does $X=$??\\

Option B:\\
\hline

Finally, \\
The accumulated value of option B$= XS_{\actuarialangle{24}{.05}}$\\
(How much Jim ends up with when Jim brings in X valued level payments for 24 years.)\\
Consequently,\\
$XS_{\actuarialangle{24}{.05}} = 10,000\left(1+i\right)^{24} $\\
OR $AV(B) =  AV(A)$\\



%$S_{\actuarialangle{24}} = 10,000\left(1+0.1\right)^{24} + 10,000\left(1+0.05\right)^{24}$\\


\end{section}
% --------------------------------------------------------------
%     You don't have to mess with anything below this line.
% --------------------------------------------------------------
 
\end{document}
