% --------------------------------------------------------------
% This is all preamble stuff that you don't have to worry about.
% Head down to where it says "Start here"
% --------------------------------------------------------------
 
\documentclass[12pt]{article}
\usepackage{actuarialsymbol}
 
\usepackage[margin=1in]{geometry} 
\usepackage{amsmath,amsthm,amssymb}

 
\newcommand{\N}{\mathbb{N}}
\newcommand{\Z}{\mathbb{Z}}
 
\newenvironment{theorem}[2][Theorem]{\begin{trivlist}
\item[\hskip \labelsep {\bfseries #1}\hskip \labelsep {\bfseries #2.}]}{\end{trivlist}}
\newenvironment{lemma}[2][Lemma]{\begin{trivlist}
\item[\hskip \labelsep {\bfseries #1}\hskip \labelsep {\bfseries #2.}]}{\end{trivlist}}
\newenvironment{exercise}[2][Exercise]{\begin{trivlist}
\item[\hskip \labelsep {\bfseries #1}\hskip \labelsep {\bfseries #2.}]}{\end{trivlist}}
\newenvironment{problem}[2][Problem]{\begin{trivlist}
\item[\hskip \labelsep {\bfseries #1}\hskip \labelsep {\bfseries #2.}]}{\end{trivlist}}
\newenvironment{question}[2][Question]{\begin{trivlist}
\item[\hskip \labelsep {\bfseries #1}\hskip \labelsep {\bfseries #2.}]}{\end{trivlist}}
\newenvironment{corollary}[2][Corollary]{\begin{trivlist}
\item[\hskip \labelsep {\bfseries #1}\hskip \labelsep {\bfseries #2.}]}{\end{trivlist}}
 
\begin{document}
 
% --------------------------------------------------------------
%                         Start here
% --------------------------------------------------------------
 
\title{Lecture 4}%replace X with the appropriate number
\author{Clark Saben\\ %replace with your name
TOI} %if necessary, replace with your course title
 
\maketitle

\section{Todos}

\begin{itemize}
	\item (tentative) Celebration of learning posted next weds due Tuesday 3/28 9am
	\item hw4 p1 and p2 due Sunday
\end{itemize}

%ideally working from a master.tex that shows previous days would be nice
\section{hw4 problem 1}
solution in green notebook

\section{2.4 Annuity Due \& Annuity-Immediate}
So far, our focus has been on annuities for which payments occur at the end of the perod, 
i.e. annuity-immediate. When payments are made at the start of each period, we speak 
of an annuity-due. In this section, we look at the PV and FV of an annuity due. For an 
annuity-immediate, \\
the PV is,\\
\begin{align*}
$$a_{\actuarialangle{n}} = \frac{1-v^n}{i}$$\\
\end{align*}
and the FV is,\\
\begin{align*}
$$S_{\actuarialangle{n}} = \frac{(1+i)^n-1}{i}$$\\
\end{align*}

Now, let us consider the situation for an annuity due: that is successive payments
of 1 are made at the beginning of each period for n periods.\\

The accumulated value of the series of payments is, GREENBOOK\\
%how to do the double dots on top of a symbol? all in page 2 of green notebook

Now, lets turn our atention to the PV of an annuity due.\\
	$$
	a^{..}_{\actuarialangle{n}i} = 1+v+\cdots+v^{n-1}=\\
	a^{..}_{\actuarialangle{n}i} = \frac{1-v^n}{1-v}=\\
	\boxed{a^{..}_{\actuarialangle{n}i} = \frac{1-v^n}{d}}\\
	$$

Remark (CN):\\
$$
a^{..}_{\actuarialangle{n}i} = a_{\actuarialangle{n}}(1+i)\\
$$
$$
S^{..}_{\actuarialangle{n}i} = S_{\actuarialangle{n}}(1+i)\\
$$
$$
%infinity symbol
a^{..}_{\actuarialangle{n->\infty}i} = \frac{1}{d} = a_{\actuarialangle{n->\infty}}(1+i)+a_{\actuarialangle{n->\infty}
}+1\\
$$
\section{2.4.1: e.g. (hw4 no.2)} 
%underline
Jim began saving for his retirement by making monthly deposits of 200 into a fund earning 6\%
and missed deposits 60-72. He then continued to make monthly deposits of 200 until December 21,
2009. How much did Jim accumulate in his account including interest on December 21, 2009?\\

option 1:\\
Accumulated value of 59 payments + Remaining $\lambda$ payments \\











 

 
% --------------------------------------------------------------
%     You don't have to mess with anything below this line.
% --------------------------------------------------------------
 
\end{document}
