% --------------------------------------------------------------
% This is all preamble stuff that you don't have to worry about.
% Head down to where it says "Start here"
% --------------------------------------------------------------
 
\documentclass[12pt]{article}
\usepackage{actuarialsymbol}
 
\usepackage[margin=1in]{geometry} 
\usepackage{amsmath,amsthm,amssymb}

 
\newcommand{\N}{\mathbb{N}}
\newcommand{\Z}{\mathbb{Z}}
 
\newenvironment{theorem}[2][Theorem]{\begin{trivlist}
\item[\hskip \labelsep {\bfseries #1}\hskip \labelsep {\bfseries #2.}]}{\end{trivlist}}
\newenvironment{lemma}[2][Lemma]{\begin{trivlist}
\item[\hskip \labelsep {\bfseries #1}\hskip \labelsep {\bfseries #2.}]}{\end{trivlist}}
\newenvironment{exercise}[2][Exercise]{\begin{trivlist}
\item[\hskip \labelsep {\bfseries #1}\hskip \labelsep {\bfseries #2.}]}{\end{trivlist}}
\newenvironment{problem}[2][Problem]{\begin{trivlist}
\item[\hskip \labelsep {\bfseries #1}\hskip \labelsep {\bfseries #2.}]}{\end{trivlist}}
\newenvironment{question}[2][Question]{\begin{trivlist}
\item[\hskip \labelsep {\bfseries #1}\hskip \labelsep {\bfseries #2.}]}{\end{trivlist}}
\newenvironment{corollary}[2][Corollary]{\begin{trivlist}
\item[\hskip \labelsep {\bfseries #1}\hskip \labelsep {\bfseries #2.}]}{\end{trivlist}}
 
\begin{document}
 
% --------------------------------------------------------------
%                         Start here
% --------------------------------------------------------------
 
\title{Lecture 11}%replace X with the appropriate number
\author{Clark Saben\\ %replace with your name
TOI} %if necessary, replace with your course title
 
\maketitle

\section{Todos}

\begin{itemize}
	\item get these notes into Charlie Cruz's notation (or someone not me lol)
	\item tex lect 7 from kevin and the earlier lect from kat
	\item today's mantra: write math first then tex in dead moments
	\item Final is take home (due wednesday 4/26). mix of open ended and multiple choice
	\item schedule a tex all lecture notes day
\end{itemize}

\section{4.2 Bonds}
\begin{enumerate}
\item face amount
\item number of coupons
\item coupon rate
\item maturity
\item coupon payment
\end{enumerate}

Price = $Fra_{\actuarialangle{n}j} + Cv_j^n$
\subsection{4.2.1}
E.g.
Calculate the price of a 5\% 10 year bond with a face amount of a 100,
semiannual coupons, and yield rate of 7.2\% compounded semiannually.
Solution:\\
$$
\begin{aligned}
	F &= 100 \\
	n &= 20 \\
	C &= 100
	r &= 0.05 \\
	j &=  \frac{7.2}{2} = 3.6\%\\
\end{aligned}
$$
Thus, GREENBOOK
$$
\begin{aligned}
	\text{Price} &= $Fra_{\actuarialangle{n}j} + Cv_j^n$
\end{aligned}

 
% --------------------------------------------------------------
%     You don't have to mess with anything below this line.
% --------------------------------------------------------------
 
\end{document}
