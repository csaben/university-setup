% --------------------------------------------------------------
% This is all preamble stuff that you don't have to worry about.
% Head down to where it says "Start here"
% --------------------------------------------------------------
<<<<<<< HEAD
=======
 
>>>>>>> ec88e8a445e98a2df91eb9dd56dd21bebfb3982f
\documentclass[12pt]{article}
 
\usepackage[margin=1in]{geometry} 
\usepackage{amsmath,amsthm,amssymb}
\usepackage{actuarialsymbol}
 
\newcommand{\N}{\mathbb{N}}
\newcommand{\Z}{\mathbb{Z}}
 
\newenvironment{theorem}[2][Theorem]{\begin{trivlist}
\item[\hskip \labelsep {\bfseries #1}\hskip \labelsep {\bfseries #2.}]}{\end{trivlist}}
\newenvironment{lemma}[2][Lemma]{\begin{trivlist}
\item[\hskip \labelsep {\bfseries #1}\hskip \labelsep {\bfseries #2.}]}{\end{trivlist}}
\newenvironment{exercise}[2][Exercise]{\begin{trivlist}
\item[\hskip \labelsep {\bfseries #1}\hskip \labelsep {\bfseries #2.}]}{\end{trivlist}}
\newenvironment{problem}[2][Problem]{\begin{trivlist}
\item[\hskip \labelsep {\bfseries #1}\hskip \labelsep {\bfseries #2.}]}{\end{trivlist}}
\newenvironment{question}[2][Question]{\begin{trivlist}
\item[\hskip \labelsep {\bfseries #1}\hskip \labelsep {\bfseries #2.}]}{\end{trivlist}}
\newenvironment{corollary}[2][Corollary]{\begin{trivlist}
\item[\hskip \labelsep {\bfseries #1}\hskip \labelsep {\bfseries #2.}]}{\end{trivlist}}
 
\begin{document}
 
% --------------------------------------------------------------
%                         Start here
% --------------------------------------------------------------
 
<<<<<<< HEAD
\title{TOI Homework}%replace X with the appropriate number
=======
\title{Theory of Interest Homework}%replace X with the appropriate number
>>>>>>> ec88e8a445e98a2df91eb9dd56dd21bebfb3982f
\author{Steven, Emma, Clark\\ %replace with your name
}% 3/6/23 lecture} %if necessary, replace with your course title
% \title{}%replace X with the appropriate number
%sample envs
% \begin{theorem}[Theorem 1.1]
% \begin{lemma}[Lemma 1.1]
% \begin{exercise}[Exercise 1.1]
% \begin{problem}[Problem 1.1]
% \begin{question}[Question 1.1]
% \begin{corollary}[Corollary 1.1]
\maketitle

<<<<<<< HEAD
% \begin{question}{Question 2.1.1}
% 	 In the preceding example, determine the amount of each monthly pay-ment if no payment is made for the first 12 months.
% \end{question}
% \begin{problem}
% 	% $$
% 	% xv (\frac{1-v^{348}}{1-v}) = 200000
% 	% $$
% \end{problem}
% \hline


=======
\begin{question}[Question 2.2.7]
	a Find the monthly payments for a 30 year fixed loan of 200,000
	with an APR of 4.5\% compounded monthly + payments made at the end
	of each month.\\

	Recall, as a FV problem we can state this as follows,\\
	\begin{align*}
		xS_{\actuarialangle{n}i} &= 200,000 \left(1+i \right)^{360}\\
	\end{align*}
	Note also, that our discount factor in this scenario is,\\
	\begin{align*}
		v &= \frac{1}{1+\frac{i}{12}}\\
		v_{0.045} &= \frac{1}{1+\frac{.045}{12}}\\
		v_{0.045} &= 0.9962\\
	\end{align*}
	Keeping all this in mind to find the monthly payments  given that we known 30 years
	is 360 months, we can use the following equivalent PV method,\\
	\begin{align*}
		xv \left(\frac{1-v^{360}}{1-v}\right) &= 200,000\\
		x \left(0.9962 \right)(\frac{1-.9962^{360}}{1-.9962})&= 200,000\\
		x \left(197.361\right) &= 200,000\\
		x &= 1013.371\\
	\end{align*}

\hline
\pagebreak

\begin{question}[Question 2.2.11]
	a In the preceeding example, determine the amount of each monthly payment
	if no payment is made for the first 12 months.\\
\end{question}

Consider a new period of 360 months - 12 months and let's use our same method from the previous question,\\
	\begin{align*}
	{xv \left(\frac{1-v^{348}}{1-v}\right)} &= 200,000\\
	x \left(0.99626 \right)(\frac{1-.99626^{348}}{1-.99626})&= 200,000\\
	x \left(194.177\right) &= 200,000\\
	x &= 1029.986\\
	\end{align*}
    
\hline

\begin{question}[Problem 2.2.13] 
a Jim can make an investment of 10,000 in 2 ways:\\  
\begin{enumerate}  
\item Deposits into an account yielding an annual interest rate of $i$.
\item He can purchase an annuity immediate (payments occur at end of month) with
24 level payments (the amounts don't change) annually, at an
annual rate of 10\%. These payments are then deposited into
a fund that yields an annual effective rate of 5\%.   
\end{enumerate}
If both options produce the same accumulated value at the end of 24 years
what is the value of $i$? 
\end{question}

We will determine the respective accumulated values and equate them to find $i$.\\

Accumlated value of option 1:\\
\begin{align*}
	10,000 \left(1+i \right)^{24} 
\end{align*}

First, let $x$ be the level payment, then $10,000 = xa_{\actuarialangle{24}.1}$, therefore,\\
\begin{align*}
	x &= \frac{10,000}{a_{\actuarialangle{24}.1}}\\
	x &= \frac{10,000}{\frac{1-v^{24}}{.1}} \text{ where i = 0.1}\\
	x &= \frac{10,000}{\frac{1-.1.015}{.1}}\\
	x &= 1,112.9978\\
\end{align*}

Finally, Accumlated value of option 2:\\
\begin{align*}
	xS_{\actuarialangle{24}.05}
\end{align*}

Consequently,\\

$$
\begin{align*}
    xS_{\actuarialangle{24}.05} &= 10,000 \left(1 + i \right)^{24}\\
    x \left( \frac{(1+i^n)-1}{i} \right) &= 10,000 \left(1 + i \right)^{24}\\
    x \left( \frac{(1+.05^{24})-1}{.05} \right) &= 10,000 \left(1 + i \right)^{24}\\
    1112.9978 \left(44.502 \right)&= 10,000 \left(1 + i \right)^{24}\\
    49530.62524 &= 10,000 \left(1 + i \right)^{24}\\
    \left(4.9530.62524 \right)^{\frac{1}{24}} &= 1 + i\\
    i &= 6.89\%\\
\end{align*}
$$
>>>>>>> ec88e8a445e98a2df91eb9dd56dd21bebfb3982f
% --------------------------------------------------------------
%     You don't have to mess with anything below this line.
% --------------------------------------------------------------
 
\end{document}
