% --------------------------------------------------------------
% This is all preamble stuff that you don't have to worry about.
% Head down to where it says "Start here"
% --------------------------------------------------------------
 
\documentclass[12pt]{article}
\usepackage{actuarialsymbol}
 
\usepackage[margin=1in]{geometry} 
\usepackage{amsmath,amsthm,amssymb}

 
\newcommand{\N}{\mathbb{N}}
\newcommand{\Z}{\mathbb{Z}}
 
\newenvironment{theorem}[2][Theorem]{\begin{trivlist}
\item[\hskip \labelsep {\bfseries #1}\hskip \labelsep {\bfseries #2.}]}{\end{trivlist}}
\newenvironment{lemma}[2][Lemma]{\begin{trivlist}
\item[\hskip \labelsep {\bfseries #1}\hskip \labelsep {\bfseries #2.}]}{\end{trivlist}}
\newenvironment{exercise}[2][Exercise]{\begin{trivlist}
\item[\hskip \labelsep {\bfseries #1}\hskip \labelsep {\bfseries #2.}]}{\end{trivlist}}
\newenvironment{problem}[2][Problem]{\begin{trivlist}
\item[\hskip \labelsep {\bfseries #1}\hskip \labelsep {\bfseries #2.}]}{\end{trivlist}}
\newenvironment{question}[2][Question]{\begin{trivlist}
\item[\hskip \labelsep {\bfseries #1}\hskip \labelsep {\bfseries #2.}]}{\end{trivlist}}
\newenvironment{corollary}[2][Corollary]{\begin{trivlist}
\item[\hskip \labelsep {\bfseries #1}\hskip \labelsep {\bfseries #2.}]}{\end{trivlist}}
 
\begin{document}
 
% --------------------------------------------------------------
%                         Start here
% --------------------------------------------------------------
 
\title{Lecture 8}%replace X with the appropriate number
\author{Clark Saben\\ %replace with your name
TOI} %if necessary, replace with your course title
 
\maketitle

\section{Todos}

\begin{itemize}
	\item HW5 (2.4.2, 2.4.3, 2.5.2, 2.5.3) due tomorrow 10am
	\item HW6 due next tuesday 10am (7 is due next next double check if i need to)
	\item hw6 q1 is that amortization schedule that I made last time
	\item get these notes into Charlie Cruz's notation (or someone not me lol)
	\item tex lect 7 from kevin and the earlier lect from kat
	\item today's mantra: write math first then tex in dead moments
\end{itemize}

\section{3.2. Amortized Loans with Level Payments}
%overwrite default section numbering
\subsection{3.2.1: Remark}
$$L = OB_{0}= a_{\actuarialangle{n}i}$$
Prospective View:
$$\begin{aligned}
	OB_t &= a_{\actuarialangle{n-t}i}
\end{aligned}$$
Retroactive View:
$$\begin{aligned}
	OB_t &= L(1+i)^{t} - S_{\actuarialangle{t}i} \\
\end{aligned}$$
From the above, we can observe that,
$$\begin{aligned}
	I_t &= a_{n-(t-1)}i = a_{\actuarialangle{n-t}i}i\\
	I_t &= ia_{n-t+1}\\
	I_t &= i \left(\frac{1-v^{n-t+1}}{i}\right)\\
	I_t &= OB_{t-1}i = 1-v^{n-t+1}\\
\end{aligned}$$

\subsection{3.2.2: Remark:}
Sometimes we like to know the total interest paid. In particular, 
$$\begin{aligned}
	I &= I_1 + I_2 + \cdots + I_n\\
	  &= OB_0i + OB_1i + \cdots + OB_{n-1}i\\
	  &= \text{re-write in terms of $a_{\actuarialangle{n}i}$}\\
	  &= a_{\actuarialangle{n}i}i + a_{\actuarialangle{n-1}i}i + \cdots + a_{\actuarialangle{n-(n-1)}i}i\\
	  %factor out i
	  &= (a_{\actuarialangle{n}} + a_{\actuarialangle{n-1}} + \cdots + a_{\actuarialangle{n-(n-1)}})i\\
	  &= (\frac{1-v^{n}}{i} + \frac{1-v^{n-1}}{i} + \cdots + \frac{1-v^{1}}{i})i\\ 
	  &= (1-v^n) + (1-v^{n-1}) + \cdots + (1-v^{1})\\
	  &= n - (v^n + v^{n-1} + \cdots + v^{1})\\
	  &= n - a_{\actuarialangle{n}}
\end{aligned}$$
\subsection{3.2.3: Example hw6 pr2 see GB:}
Sam borrows $X$ for 10 years at an annual effective rate of 8\%. 
If he pays the principal and the accumulated interest in one lump
sum at the end of 10 years, he would pay 468.05 more in interest
than if he repaid the loan with ten level payments at the end of each year. 
Calculate $X$.

\subsection{3.2.4: Example hw6 pr3 see GB:}
A loan L is to be repaid with 40 payments of 100 at the end of each quarter. Interest
on the loan is charged at a nominal rate $i$, where $0<i<1$, convertible monthly.
The outstanding principlals immediately after the 8th and 24th payments
are 2308.15 and 1345.50, respectively. Calculate the amount of interest
repaid in the 15th payment.

\subsection{3.2.5: Example hw6 pr4 see GB:}
A loan at a nominal rate of 24\% convertible monthly is to be repaid with equal payments at the
end of each month for $2n$ months. The nth payment consists of equal amounts of interest and
principal. Calculate $n$.



 
% --------------------------------------------------------------
%     You don't have to mess with anything below this line.
% --------------------------------------------------------------
 
\end{document}
