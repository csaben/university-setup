% --------------------------------------------------------------
% This is all preamble stuff that you don't have to worry about.
% Head down to where it says "Start here"
% --------------------------------------------------------------
 
\documentclass[12pt]{article}
\usepackage{actuarialsymbol}
 
\usepackage[margin=1in]{geometry} 
\usepackage{amsmath,amsthm,amssymb}

 
\newcommand{\N}{\mathbb{N}}
\newcommand{\Z}{\mathbb{Z}}
 
\newenvironment{theorem}[2][Theorem]{\begin{trivlist}
\item[\hskip \labelsep {\bfseries #1}\hskip \labelsep {\bfseries #2.}]}{\end{trivlist}}
\newenvironment{lemma}[2][Lemma]{\begin{trivlist}
\item[\hskip \labelsep {\bfseries #1}\hskip \labelsep {\bfseries #2.}]}{\end{trivlist}}
\newenvironment{exercise}[2][Exercise]{\begin{trivlist}
\item[\hskip \labelsep {\bfseries #1}\hskip \labelsep {\bfseries #2.}]}{\end{trivlist}}
\newenvironment{problem}[2][Problem]{\begin{trivlist}
\item[\hskip \labelsep {\bfseries #1}\hskip \labelsep {\bfseries #2.}]}{\end{trivlist}}
\newenvironment{question}[2][Question]{\begin{trivlist}
\item[\hskip \labelsep {\bfseries #1}\hskip \labelsep {\bfseries #2.}]}{\end{trivlist}}
\newenvironment{corollary}[2][Corollary]{\begin{trivlist}
\item[\hskip \labelsep {\bfseries #1}\hskip \labelsep {\bfseries #2.}]}{\end{trivlist}}
 
\begin{document}
 
% --------------------------------------------------------------
%                         Start here
% --------------------------------------------------------------
 
\title{Lecture 6}%replace X with the appropriate number
\author{Clark Saben\\ %replace with your name
TOI} %if necessary, replace with your course title
 
\maketitle

\section{Todos}

\begin{itemize}
	\item QUIZ Celebration of learning posted next weds DUE Tuesday 3/28 9am
	\item HW5 due sometime after C2
	\item get these notes into Charlie Cruz's notation (or someone not me lol)
\end{itemize}

%ideally working from a master.tex that shows previous days would be nice
\section{}
Remark: For an increasing annuity,\\
$$
PV = \left(Ia_{\actuarialangle{n}}\right) = \frac{a^{..}_{\actuarialangle{n} -nv^n}}{i}
$$
$$
FV = \left(IS_{\actuarialangle{n}}\right) = \frac{n - a_{\actuarialangle{n}}}{i}
$$

for a Decreasing  annuity,\\
$$
PV = \left(Da_{\actuarialangle{n}}\right) = \frac{n(i+1)^n - S_{\actuarialangle{n}}}{i}
$$
$$
FV = \left(DS_{\actuarialangle{n}}\right) = \frac{S^{..}_{\actuarialangle{n} -nv^n}}{i}
$$

\section{2.5.2 Example == HW prob 3 of PSET 5:}
A five year annuity has increasing monthly payment at the end of each month. The first payment
us 600, and each subsequent payment is 10 learger than the previous payment. At a rate of
$0.5\%$ per month, find the PV of the annuity valued one month before the final payment.

Soln: There are at least two ways of approaching this problem. \\

\begin{enumerate}
	\item We can think of the 5 year annuity as a level annuity of [BLANK] per month,
		and an increasing annuity annuity with additional payments of 10.
		$$
		PV = 600a_{\actuarialangle{60}i} + 10 \left(Ia_{\actuarialangle{59}i}\right)v
		$$
		$$
		PV = 590a_{\actuarialangle{60}i} + 10 \left(Ia_{\actuarialangle{60}i}\right)
		$$
	\item We can consider the annuity as a combination of level payments of 1200, and 
		decreasing payments starting with [BLANK=600] and going down by [BLANK=10] each month.\\
		In this case, \\
		$$
		PV = 1200a_{\actuarialangle{60}i} - 10 \left(Da_{\actuarialangle{60}i}\right)
		$$


\end{enumerate}

\section{2.5.3 Example == HW Problem 4 and last of PSET 5:}
Jeff bought an increasing perpetuity-due (annuity due means its due at beginning of month
immediate is at the end of the month) with annual payments starting at 5 and increasing by 5 
each year until the payment amount reaches 100. Thereafter, the payments then remain
at 100. If the annual effective rate is $7.5\%$, what is the PV of this perpetuity?

notes: 
\begin{itemize}
	\item  $Ia_{\infty}$ will be used (but at the period when this happens you need to discount it)
		$$
		Ia_{\actuarialangle{\infty}}v^{x}
		$$
	\item  our previous prob is almost same (different periods an such)
\end{itemize}

\section{Chapter 3: Loan Repayment}
note: excel sheets can handle a lot of this
\subsection{3.1: Amortization:}
see green book for numberline
\subsection{3.1.1 Definition:}
A loan L is amortized at interest $i$ if the loan amount is equal to the PV of all the 
loan payments. $K_1, K_2, \dots, K_n$. In other words (IOW),\\
$$
L = K_1v + K_2v^2 + \dots + K_nv^n
$$
The outstanding balance of the loan at time t, denoted $OB_t$, is simply the sum
of the unpaid principal and interest at time $t$. IOW $OB_t$ is simply the
PV of the remaining payments.\\
For example, $OB_0 = L$. and $OB_n = 0$
%align env makes sense here
$$
OB_1 = L(1+i) - K_1 \text{ (you have to accumulate the interest since t=0)}\\
$$
$$
OB_2 = L(1+i)^2 - K_1 - K_2 = OB_1(1+i) - K_2 = [L(1+i) - K_1](1+i) - K_2)]
$$
$$
OB_2 = L(1+i)^2 - K_1(1+i) - K_2
$$
$$
OB_3 = L(1+i)^3 - K_1(1+i)^2 - K_2(1+i) - K_3
$$
In general, 
$$
OB_t = L(1+i)^t - \sum_{j=1}^{t}K_j(1+i)^{t-j}
$$
Or,
$$
OB_t = L(1+i)^t - K_1(1+i)^{t-1} - K_2(1+i)^{t-2} - \dots - K_{t-1}(1+i)^1 - K_{t-0}
$$
This is known as the retrospective view of $OB_t$.\\
On the other hand, $OB_t$ can also be viewed in terms of the remaining loan
repayments that is,
$$
OB_t = \text{PV of all remaining payments}
$$
$$
OB_t = K_t + K_{t+1}v + K_{t+2}v^2 + \dots + K_n v^{n-t}
$$
This is called the prospective view of $OB_t$.\\




 
% --------------------------------------------------------------
%     You don't have to mess with anything below this line.
% --------------------------------------------------------------
 
\end{document}
