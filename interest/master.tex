\documentclass[12pt]{article}
\usepackage[margin=1in]{geometry} 
\usepackage{amsmath,amsthm,amssymb}
\usepackage{actuarialsymbol}

\newcommand{\N}{\mathbb{N}}
\newcommand{\Z}{\mathbb{Z}}
 
\newenvironment{theorem}[2][Theorem]{\begin{trivlist}
\item[\hskip \labelsep {\bfseries #1}\hskip \labelsep {\bfseries #2.}]}{\end{trivlist}}
\newenvironment{lemma}[2][Lemma]{\begin{trivlist}
\item[\hskip \labelsep {\bfseries #1}\hskip \labelsep {\bfseries #2.}]}{\end{trivlist}}
\newenvironment{exercise}[2][Exercise]{\begin{trivlist}
\item[\hskip \labelsep {\bfseries #1}\hskip \labelsep {\bfseries #2.}]}{\end{trivlist}}
\newenvironment{problem}[2][Problem]{\begin{trivlist}
\item[\hskip \labelsep {\bfseries #1}\hskip \labelsep {\bfseries #2.}]}{\end{trivlist}}
\newenvironment{question}[2][Question]{\begin{trivlist}
\item[\hskip \labelsep {\bfseries #1}\hskip \labelsep {\bfseries #2.}]}{\end{trivlist}}
\newenvironment{corollary}[2][Corollary]{\begin{trivlist}
\item[\hskip \labelsep {\bfseries #1}\hskip \labelsep {\bfseries #2.}]}{\end{trivlist}}

\title{Theory of Interest}
\begin{document}
\maketitle
%2,3,4,6,8,9,10,11
	\tableofcontents
	%start lectures
	% % --------------------------------------------------------------
% This is all preamble stuff that you don't have to worry about.
% Head down to where it says "Start here"
% --------------------------------------------------------------
 
\documentclass[12pt]{article}
 
\usepackage[margin=1in]{geometry} 
\usepackage{amsmath,amsthm,amssymb}
\usepackage{actuarialsymbol}
 
\newcommand{\N}{\mathbb{N}}
\newcommand{\Z}{\mathbb{Z}}
 
\newenvironment{theorem}[2][Theorem]{\begin{trivlist}
\item[\hskip \labelsep {\bfseries #1}\hskip \labelsep {\bfseries #2.}]}{\end{trivlist}}
\newenvironment{lemma}[2][Lemma]{\begin{trivlist}
\item[\hskip \labelsep {\bfseries #1}\hskip \labelsep {\bfseries #2.}]}{\end{trivlist}}
\newenvironment{exercise}[2][Exercise]{\begin{trivlist}
\item[\hskip \labelsep {\bfseries #1}\hskip \labelsep {\bfseries #2.}]}{\end{trivlist}}
\newenvironment{problem}[2][Problem]{\begin{trivlist}
\item[\hskip \labelsep {\bfseries #1}\hskip \labelsep {\bfseries #2.}]}{\end{trivlist}}
\newenvironment{question}[2][Question]{\begin{trivlist}
\item[\hskip \labelsep {\bfseries #1}\hskip \labelsep {\bfseries #2.}]}{\end{trivlist}}
\newenvironment{corollary}[2][Corollary]{\begin{trivlist}
\item[\hskip \labelsep {\bfseries #1}\hskip \labelsep {\bfseries #2.}]}{\end{trivlist}}
 
\begin{document}
 
% --------------------------------------------------------------
%                         Start here
% --------------------------------------------------------------
 
\title{Theory of Interest}%replace X with the appropriate number
\author{Clark Saben\\ %replace with your name
}% 3/6/23 lecture} %if necessary, replace with your course title
% \title{}%replace X with the appropriate number
%sample envs
% \begin{theorem}[Theorem 1.1]
% \begin{lemma}[Lemma 1.1]
% \begin{exercise}[Exercise 1.1]
% \begin{problem}[Problem 1.1]
% \begin{question}[Question 1.1]
% \begin{corollary}[Corollary 1.1]
\maketitle

\begin{section}{3/6/23 Lecture}

%iitemize
Homework and Logistics
\hline 
\begin{itemize}
	\item wednesday 4pm math meeting
	\item illustrations are in green small handbook associated with this day
	\item Get a credit card plan
	\item get a rider such that after your life insurance expires it keeps building up.
		you can have a rider to pull money out 60\% if you are terminally ill. having a 
		will makes things better.
	\item Make a master tex file for this folder
	\item fix errors and make equations look nice

\end{itemize}
\hline

\begin{question}[Example 2.2.6]
	a consider and annuity as in e.g. 2.2.4 with the following adjustments. 
	suppose that the interest rate is $12\%$ per annun for the first 10
	months with payments of $X$ each, and the rate doubles to $24\%$ for
	the rest of the term with payments of $2X$ each. Determine the level
	payment for each period.
\end{question}

\boxed{Solution:}\\
Recall that the accumulated value of this annuity (see e.g. 2.2.4) is 10000,
consequently (keep in mind we accumulate the 10 years), 
\begin{align*}
	%busted because I cannot find actuarial angle
	{XS_{\actuarialangle{10}i_{1}}\left(1+i_{2}\right)^{10} + 2XS_{\actuarialangle{10}i_{2}}} &= 10000 \\
	X  \frac{{}[\left(1.01)^{10} -1]}{0.01} + 2X \frac{{}[\left(1.02)^{10} -1]}{0.02} &= 10000 \\
	X &= 288.58
	% X\left
\end{align*}

\pagebreak

\begin{question}[Example 2.2.7]
	b Find the monthly payment for a 30 year fixed loan of 200,000 with 
	APR of 4.5\% compounded monthly, and payments made at the end of each month.
\end{question}

\boxed{Solution:}\\

This problem can be approached in two ways:\\

\begin{enumerate}
	\item  Future value of a loan (number lie from 1 to 360 where loan is given at 1).\\
	X(Accum Value) should = Accum value of the Loan\\
	$XS\actuarialangle{n}i = 200,000(1+i)$\\
\item Present value of a loan (number lie from 1 to 360 where loan is given at 360).\\ 
	PV discussion happens at time=0
\end{enumerate}

\hline

%new sign
Let $v$ be the discount factor, where $v = \frac{1}{1+i}$, and $i$ is the interest rate.\\
Then the sum of all discounts should = the PV of the loan.\\

%make this prettier
$Xv + Xv^2+ ... + Xv^{360} = 200,000$\\
$Xv\left(1+v+v^2+...+v^{359}\right) = 200,000$ \text{(notice we have the geometriic series)}\\
$Xv\frac{[1-v^{360}]}{[1-v]} = 200,000$\\

Now,  find $X$\\


2.2.8 Remark:
In general, we denote\\
$v + v^2 +..+ v^n = a\actuarialangle{n}i$, where $v=\frac{1}{1+i}$\\

2.2.9 Definition: \\
$a\actuarialangle{n}i$ is the present value of an annuity-immediate with level payments of 1 occuring 
at the end of each month for n months at a rate of $i$ where a\actuarialangle{n}i = v+ v^2 +..+ v^n$\\

\begin{theorem}[Theorem 2.2.10]
	1  a_{\actuarialangle{n}i} = v+v^2+...+v^n = \frac{1-v^{n}}{i}\\
\end{theorem}

%align this
\begin{proof}
	Observe that,\\
	$a_{\actuarialangle{n}i} &= v+v^2+...+v^n = v[1+v+v^2+...+v^{n-1}]$\\
	$ &= v\frac{[1-v^{n}]}{[1-v]}$\\

	It suffices to show that $\frac{v}{1-v} = \frac{1^}{i}$\\

	To this end, since $v = \frac{1}{1+i}$, we have\\
	$1-v = 1 - \frac{1}{1+i} = \frac{1+i}{1+i} - \frac{1}{1+i} = \frac{i}{1+i}$\\
	$\frac{v}{1-v} = \frac{1}{1+i} / \frac{i}{1+i} = \frac{1+i}{i} * \frac{1}{1+i} = \frac{1}{i}\\
	$a_{\actuarialangle{n}i} = \frac{v}{1-v^n}\left(1-v^{n}\right) = \frac{1}{i}\left(1-v^{n}\right) = \frac{1-v^n}{i}
\end{proof}

\begin{problem}[2.1.11 Example]
	a In the preceding example, determine the amount of each monthly payment if no payment
	is made for the first 12 months.
    
\end{problem}





\end{section}
% --------------------------------------------------------------
%     You don't have to mess with anything below this line.
% --------------------------------------------------------------
 
\end{document}

	% % --------------------------------------------------------------
% This is all preamble stuff that you don't have to worry about.
% Head down to where it says "Start here"
% --------------------------------------------------------------
 
\documentclass[12pt]{article}
 
\usepackage[margin=1in]{geometry} 
\usepackage{amsmath,amsthm,amssymb}
\usepackage{actuarialsymbol}
 
\newcommand{\N}{\mathbb{N}}
\newcommand{\Z}{\mathbb{Z}}
 
\newenvironment{theorem}[2][Theorem]{\begin{trivlist}
\item[\hskip \labelsep {\bfseries #1}\hskip \labelsep {\bfseries #2.}]}{\end{trivlist}}
\newenvironment{lemma}[2][Lemma]{\begin{trivlist}
\item[\hskip \labelsep {\bfseries #1}\hskip \labelsep {\bfseries #2.}]}{\end{trivlist}}
\newenvironment{exercise}[2][Exercise]{\begin{trivlist}
\item[\hskip \labelsep {\bfseries #1}\hskip \labelsep {\bfseries #2.}]}{\end{trivlist}}
\newenvironment{problem}[2][Problem]{\begin{trivlist}
\item[\hskip \labelsep {\bfseries #1}\hskip \labelsep {\bfseries #2.}]}{\end{trivlist}}
\newenvironment{question}[2][Question]{\begin{trivlist}
\item[\hskip \labelsep {\bfseries #1}\hskip \labelsep {\bfseries #2.}]}{\end{trivlist}}
\newenvironment{corollary}[2][Corollary]{\begin{trivlist}
\item[\hskip \labelsep {\bfseries #1}\hskip \labelsep {\bfseries #2.}]}{\end{trivlist}}
 
\begin{document}
 
% --------------------------------------------------------------
%                         Start here
% --------------------------------------------------------------
 
\title{Theory of Interest}%replace X with the appropriate number
\author{Clark Saben\\ %replace with your name
}% 3/6/23 lecture} %if necessary, replace with your course title
% \title{}%replace X with the appropriate number
%sample envs
% \begin{theorem}[Theorem 1.1]
% \begin{lemma}[Lemma 1.1]
% \begin{exercise}[Exercise 1.1]
% \begin{problem}[Problem 1.1]
% \begin{question}[Question 1.1]
% \begin{corollary}[Corollary 1.1]
\maketitle

\begin{section}{3/8/23 Lecture}

%iitemize
Homework and Logistics
\hline 
\begin{itemize}
	\item 3:30pm talk  (student engagement)
	\item PS3: 2.2.7, 2.2.11, 2.2.13 (first two on last lecture notes, 3rd on today), 
		Due: 6pm on Tuesday 3/14 with an extra problem given on monday. In total,
		4 problems.
\end{itemize}
\hline

Remark:\\

In general, \\
$ S_{\actuarialangle{n}
$ a_{\actuarialangle{n}}:=$ is the present value of "    ".\\
Thus,\\
$v^{k}a_{\actuarialangle{n}}:= a_{\actuarialangle{n+k}} - a_{\actuarialangle{k}}$\\

Imagine a number line with 0 and $k$ and $k+1$	on it.\

Therefore,\\
$a_{\actuarialangle{n+k}} = a_{\actuarialangle{k}} + v^{k}a_{\actuarialangle{n}}$\\

As we saw in the last lecture,\\
$S_{\actuarialangle{n+k}} = S_{\actuarialangle{n}}\left(1+i\right)^{k} + S_{\actuarialangle{n}}$.\\
Here,\\
$v^{k}a_{\actuarialangle{n}}$ is the PV of a k period deferred annuity immediate.\\

\begin{question}[Example 2.2.13]
	a Jim can make an investment of 10,000 in 2 ways:\\
	\begin{enumerate}
		\item Deposits into an account yielding an annual interest rate of $i$.
		\item He can purchase an annuity immediate (payments occur at end of month) with
			24 level payments (the amounts don't change) annually, at an 
			annual rate of 10\%. These payments are then deposited into
			a fund that yields an annual effective rate of 5\%. 
	\end{enumerate}
	If both options produce the same accumulated value at the end of 24 years
	what is the value of $i$?
\end{question}

\boxed{Solution:}\\


Option A:\\
\hline

We will determine the respective accumuluated values. And equate them, to find $i$.\\

Accumulated value of the first option:\\
$S_{\actuarialangle{24}} = 10,000\left(1+i\right)^{24}$\\

Accumulated value of the second option:\\

Let $X$ be the level payment.\\
Then, \\
$10,000 = Xa_{\actuarialangle{24}.1}$\\
Therefore,\\
$X = \frac{10,000}{a_{\actuarialangle{24}.1}} = \frac{10000}{\frac{1-v^{24}}{i}}$ \\
$i=10$ in the actuarial angle\%\\
What does $X=$??\\

Option B:\\
\hline

Finally, \\
The accumulated value of option B$= XS_{\actuarialangle{24}{.05}}$\\
(How much Jim ends up with when Jim brings in X valued level payments for 24 years.)\\
Consequently,\\
$XS_{\actuarialangle{24}{.05}} = 10,000\left(1+i\right)^{24} $\\
OR $AV(B) =  AV(A)$\\



%$S_{\actuarialangle{24}} = 10,000\left(1+0.1\right)^{24} + 10,000\left(1+0.05\right)^{24}$\\


\end{section}
% --------------------------------------------------------------
%     You don't have to mess with anything below this line.
% --------------------------------------------------------------
 
\end{document}

	% \input{lecture3.tex}
	% % --------------------------------------------------------------
% This is all preamble stuff that you don't have to worry about.
% Head down to where it says "Start here"
% --------------------------------------------------------------
 
\documentclass[12pt]{article}
\usepackage{actuarialsymbol}
 
\usepackage[margin=1in]{geometry} 
\usepackage{amsmath,amsthm,amssymb}

 
\newcommand{\N}{\mathbb{N}}
\newcommand{\Z}{\mathbb{Z}}
 
\newenvironment{theorem}[2][Theorem]{\begin{trivlist}
\item[\hskip \labelsep {\bfseries #1}\hskip \labelsep {\bfseries #2.}]}{\end{trivlist}}
\newenvironment{lemma}[2][Lemma]{\begin{trivlist}
\item[\hskip \labelsep {\bfseries #1}\hskip \labelsep {\bfseries #2.}]}{\end{trivlist}}
\newenvironment{exercise}[2][Exercise]{\begin{trivlist}
\item[\hskip \labelsep {\bfseries #1}\hskip \labelsep {\bfseries #2.}]}{\end{trivlist}}
\newenvironment{problem}[2][Problem]{\begin{trivlist}
\item[\hskip \labelsep {\bfseries #1}\hskip \labelsep {\bfseries #2.}]}{\end{trivlist}}
\newenvironment{question}[2][Question]{\begin{trivlist}
\item[\hskip \labelsep {\bfseries #1}\hskip \labelsep {\bfseries #2.}]}{\end{trivlist}}
\newenvironment{corollary}[2][Corollary]{\begin{trivlist}
\item[\hskip \labelsep {\bfseries #1}\hskip \labelsep {\bfseries #2.}]}{\end{trivlist}}
 
\begin{document}
 
% --------------------------------------------------------------
%                         Start here
% --------------------------------------------------------------
 
\title{Lecture 4}%replace X with the appropriate number
\author{Clark Saben\\ %replace with your name
TOI} %if necessary, replace with your course title
 
\maketitle

\section{Todos}

\begin{itemize}
	\item (tentative) Celebration of learning posted next weds due Tuesday 3/28 9am
	\item hw4 p1 and p2 due Sunday
\end{itemize}

%ideally working from a master.tex that shows previous days would be nice
\section{hw4 problem 1}
solution in green notebook

\section{2.4 Annuity Due \& Annuity-Immediate}
So far, our focus has been on annuities for which payments occur at the end of the perod, 
i.e. annuity-immediate. When payments are made at the start of each period, we speak 
of an annuity-due. In this section, we look at the PV and FV of an annuity due. For an 
annuity-immediate, \\
the PV is,\\
\begin{align*}
$$a_{\actuarialangle{n}} = \frac{1-v^n}{i}$$\\
\end{align*}
and the FV is,\\
\begin{align*}
$$S_{\actuarialangle{n}} = \frac{(1+i)^n-1}{i}$$\\
\end{align*}

Now, let us consider the situation for an annuity due: that is successive payments
of 1 are made at the beginning of each period for n periods.\\

The accumulated value of the series of payments is, GREENBOOK\\
%how to do the double dots on top of a symbol? all in page 2 of green notebook

Now, lets turn our atention to the PV of an annuity due.\\
	$$
	a^{..}_{\actuarialangle{n}i} = 1+v+\cdots+v^{n-1}=\\
	a^{..}_{\actuarialangle{n}i} = \frac{1-v^n}{1-v}=\\
	\boxed{a^{..}_{\actuarialangle{n}i} = \frac{1-v^n}{d}}\\
	$$

Remark (CN):\\
$$
a^{..}_{\actuarialangle{n}i} = a_{\actuarialangle{n}}(1+i)\\
$$
$$
S^{..}_{\actuarialangle{n}i} = S_{\actuarialangle{n}}(1+i)\\
$$
$$
%infinity symbol
a^{..}_{\actuarialangle{n->\infty}i} = \frac{1}{d} = a_{\actuarialangle{n->\infty}}(1+i)+a_{\actuarialangle{n->\infty}
}+1\\
$$
\section{2.4.1: e.g. (hw4 no.2)} 
%underline
Jim began saving for his retirement by making monthly deposits of 200 into a fund earning 6\%
and missed deposits 60-72. He then continued to make monthly deposits of 200 until December 21,
2009. How much did Jim accumulate in his account including interest on December 21, 2009?\\

option 1:\\
Accumulated value of 59 payments + Remaining $\lambda$ payments \\











 

 
% --------------------------------------------------------------
%     You don't have to mess with anything below this line.
% --------------------------------------------------------------
 
\end{document}

	% % --------------------------------------------------------------
% This is all preamble stuff that you don't have to worry about.
% Head down to where it says "Start here"
% --------------------------------------------------------------
 
\documentclass[12pt]{article}
\usepackage{actuarialsymbol}
 
\usepackage[margin=1in]{geometry} 
\usepackage{amsmath,amsthm,amssymb}

 
\newcommand{\N}{\mathbb{N}}
\newcommand{\Z}{\mathbb{Z}}
 
\newenvironment{theorem}[2][Theorem]{\begin{trivlist}
\item[\hskip \labelsep {\bfseries #1}\hskip \labelsep {\bfseries #2.}]}{\end{trivlist}}
\newenvironment{lemma}[2][Lemma]{\begin{trivlist}
\item[\hskip \labelsep {\bfseries #1}\hskip \labelsep {\bfseries #2.}]}{\end{trivlist}}
\newenvironment{exercise}[2][Exercise]{\begin{trivlist}
\item[\hskip \labelsep {\bfseries #1}\hskip \labelsep {\bfseries #2.}]}{\end{trivlist}}
\newenvironment{problem}[2][Problem]{\begin{trivlist}
\item[\hskip \labelsep {\bfseries #1}\hskip \labelsep {\bfseries #2.}]}{\end{trivlist}}
\newenvironment{question}[2][Question]{\begin{trivlist}
\item[\hskip \labelsep {\bfseries #1}\hskip \labelsep {\bfseries #2.}]}{\end{trivlist}}
\newenvironment{corollary}[2][Corollary]{\begin{trivlist}
\item[\hskip \labelsep {\bfseries #1}\hskip \labelsep {\bfseries #2.}]}{\end{trivlist}}
 
\begin{document}
 
% --------------------------------------------------------------
%                         Start here
% --------------------------------------------------------------
 
\title{Lecture 6}%replace X with the appropriate number
\author{Clark Saben\\ %replace with your name
TOI} %if necessary, replace with your course title
 
\maketitle

\section{Todos}

\begin{itemize}
	\item QUIZ Celebration of learning posted next weds DUE Tuesday 3/28 9am
	\item HW5 due sometime after C2
	\item get these notes into Charlie Cruz's notation (or someone not me lol)
\end{itemize}

%ideally working from a master.tex that shows previous days would be nice
\section{}
Remark: For an increasing annuity,\\
$$
PV = \left(Ia_{\actuarialangle{n}}\right) = \frac{a^{..}_{\actuarialangle{n} -nv^n}}{i}
$$
$$
FV = \left(IS_{\actuarialangle{n}}\right) = \frac{n - a_{\actuarialangle{n}}}{i}
$$

for a Decreasing  annuity,\\
$$
PV = \left(Da_{\actuarialangle{n}}\right) = \frac{n(i+1)^n - S_{\actuarialangle{n}}}{i}
$$
$$
FV = \left(DS_{\actuarialangle{n}}\right) = \frac{S^{..}_{\actuarialangle{n} -nv^n}}{i}
$$

\section{2.5.2 Example == HW prob 3 of PSET 5:}
A five year annuity has increasing monthly payment at the end of each month. The first payment
us 600, and each subsequent payment is 10 learger than the previous payment. At a rate of
$0.5\%$ per month, find the PV of the annuity valued one month before the final payment.

Soln: There are at least two ways of approaching this problem. \\

\begin{enumerate}
	\item We can think of the 5 year annuity as a level annuity of [BLANK] per month,
		and an increasing annuity annuity with additional payments of 10.
		$$
		PV = 600a_{\actuarialangle{60}i} + 10 \left(Ia_{\actuarialangle{59}i}\right)v
		$$
		$$
		PV = 590a_{\actuarialangle{60}i} + 10 \left(Ia_{\actuarialangle{60}i}\right)
		$$
	\item We can consider the annuity as a combination of level payments of 1200, and 
		decreasing payments starting with [BLANK=600] and going down by [BLANK=10] each month.\\
		In this case, \\
		$$
		PV = 1200a_{\actuarialangle{60}i} - 10 \left(Da_{\actuarialangle{60}i}\right)
		$$


\end{enumerate}

\section{2.5.3 Example == HW Problem 4 and last of PSET 5:}
Jeff bought an increasing perpetuity-due (annuity due means its due at beginning of month
immediate is at the end of the month) with annual payments starting at 5 and increasing by 5 
each year until the payment amount reaches 100. Thereafter, the payments then remain
at 100. If the annual effective rate is $7.5\%$, what is the PV of this perpetuity?

notes: 
\begin{itemize}
	\item  $Ia_{\infty}$ will be used (but at the period when this happens you need to discount it)
		$$
		Ia_{\actuarialangle{\infty}}v^{x}
		$$
	\item  our previous prob is almost same (different periods an such)
\end{itemize}

\section{Chapter 3: Loan Repayment}
note: excel sheets can handle a lot of this
\subsection{3.1: Amortization:}
see green book for numberline
\subsection{3.1.1 Definition:}
A loan L is amortized at interest $i$ if the loan amount is equal to the PV of all the 
loan payments. $K_1, K_2, \dots, K_n$. In other words (IOW),\\
$$
L = K_1v + K_2v^2 + \dots + K_nv^n
$$
The outstanding balance of the loan at time t, denoted $OB_t$, is simply the sum
of the unpaid principal and interest at time $t$. IOW $OB_t$ is simply the
PV of the remaining payments.\\
For example, $OB_0 = L$. and $OB_n = 0$
%align env makes sense here
$$
OB_1 = L(1+i) - K_1 \text{ (you have to accumulate the interest since t=0)}\\
$$
$$
OB_2 = L(1+i)^2 - K_1 - K_2 = OB_1(1+i) - K_2 = [L(1+i) - K_1](1+i) - K_2)]
$$
$$
OB_2 = L(1+i)^2 - K_1(1+i) - K_2
$$
$$
OB_3 = L(1+i)^3 - K_1(1+i)^2 - K_2(1+i) - K_3
$$
In general, 
$$
OB_t = L(1+i)^t - \sum_{j=1}^{t}K_j(1+i)^{t-j}
$$
Or,
$$
OB_t = L(1+i)^t - K_1(1+i)^{t-1} - K_2(1+i)^{t-2} - \dots - K_{t-1}(1+i)^1 - K_{t-0}
$$
This is known as the retrospective view of $OB_t$.\\
On the other hand, $OB_t$ can also be viewed in terms of the remaining loan
repayments that is,
$$
OB_t = \text{PV of all remaining payments}
$$
$$
OB_t = K_t + K_{t+1}v + K_{t+2}v^2 + \dots + K_n v^{n-t}
$$
This is called the prospective view of $OB_t$.\\




 
% --------------------------------------------------------------
%     You don't have to mess with anything below this line.
% --------------------------------------------------------------
 
\end{document}

	% % --------------------------------------------------------------
% This is all preamble stuff that you don't have to worry about.
% Head down to where it says "Start here"
% --------------------------------------------------------------
 
\documentclass[12pt]{article}
\usepackage{actuarialsymbol}
 
\usepackage[margin=1in]{geometry} 
\usepackage{amsmath,amsthm,amssymb}

 
\newcommand{\N}{\mathbb{N}}
\newcommand{\Z}{\mathbb{Z}}
 
\newenvironment{theorem}[2][Theorem]{\begin{trivlist}
\item[\hskip \labelsep {\bfseries #1}\hskip \labelsep {\bfseries #2.}]}{\end{trivlist}}
\newenvironment{lemma}[2][Lemma]{\begin{trivlist}
\item[\hskip \labelsep {\bfseries #1}\hskip \labelsep {\bfseries #2.}]}{\end{trivlist}}
\newenvironment{exercise}[2][Exercise]{\begin{trivlist}
\item[\hskip \labelsep {\bfseries #1}\hskip \labelsep {\bfseries #2.}]}{\end{trivlist}}
\newenvironment{problem}[2][Problem]{\begin{trivlist}
\item[\hskip \labelsep {\bfseries #1}\hskip \labelsep {\bfseries #2.}]}{\end{trivlist}}
\newenvironment{question}[2][Question]{\begin{trivlist}
\item[\hskip \labelsep {\bfseries #1}\hskip \labelsep {\bfseries #2.}]}{\end{trivlist}}
\newenvironment{corollary}[2][Corollary]{\begin{trivlist}
\item[\hskip \labelsep {\bfseries #1}\hskip \labelsep {\bfseries #2.}]}{\end{trivlist}}
 
\begin{document}
 
% --------------------------------------------------------------
%                         Start here
% --------------------------------------------------------------
 
\title{Lecture 8}%replace X with the appropriate number
\author{Clark Saben\\ %replace with your name
TOI} %if necessary, replace with your course title
 
\maketitle

\section{Todos}

\begin{itemize}
	\item HW5 (2.4.2, 2.4.3, 2.5.2, 2.5.3) due tuesday 10am
	\item get these notes into Charlie Cruz's notation (or someone not me lol)
	\item tex lect 7 from kevin and the earlier lect from kat
	\item today's mantra: write math first then tex in dead moments
\end{itemize}

%ideally working from a master.tex that shows previous days would be nice
\section{3.1.3 Example}
Paul takes out a 10 year loan to be repaid by ten annual checks at the end of each year.
The first loan repayment is $X$ and each subsequent repayment is 10.16\% greater than
the previous. The annual rate on the loan is 8\%. If the amount of interest in the first
loan repayment is 892.20, find $X$. \\

\emph{Solution}:




 
% --------------------------------------------------------------
%     You don't have to mess with anything below this line.
% --------------------------------------------------------------
 
\end{document}

	% % --------------------------------------------------------------
% This is all preamble stuff that you don't have to worry about.
% Head down to where it says "Start here"
% --------------------------------------------------------------
 
\documentclass[12pt]{article}
\usepackage{actuarialsymbol}
 
\usepackage[margin=1in]{geometry} 
\usepackage{amsmath,amsthm,amssymb}

 
\newcommand{\N}{\mathbb{N}}
\newcommand{\Z}{\mathbb{Z}}
 
\newenvironment{theorem}[2][Theorem]{\begin{trivlist}
\item[\hskip \labelsep {\bfseries #1}\hskip \labelsep {\bfseries #2.}]}{\end{trivlist}}
\newenvironment{lemma}[2][Lemma]{\begin{trivlist}
\item[\hskip \labelsep {\bfseries #1}\hskip \labelsep {\bfseries #2.}]}{\end{trivlist}}
\newenvironment{exercise}[2][Exercise]{\begin{trivlist}
\item[\hskip \labelsep {\bfseries #1}\hskip \labelsep {\bfseries #2.}]}{\end{trivlist}}
\newenvironment{problem}[2][Problem]{\begin{trivlist}
\item[\hskip \labelsep {\bfseries #1}\hskip \labelsep {\bfseries #2.}]}{\end{trivlist}}
\newenvironment{question}[2][Question]{\begin{trivlist}
\item[\hskip \labelsep {\bfseries #1}\hskip \labelsep {\bfseries #2.}]}{\end{trivlist}}
\newenvironment{corollary}[2][Corollary]{\begin{trivlist}
\item[\hskip \labelsep {\bfseries #1}\hskip \labelsep {\bfseries #2.}]}{\end{trivlist}}
 
\begin{document}
 
% --------------------------------------------------------------
%                         Start here
% --------------------------------------------------------------
 
\title{Lecture 8}%replace X with the appropriate number
\author{Clark Saben\\ %replace with your name
TOI} %if necessary, replace with your course title
 
\maketitle

\section{Todos}

\begin{itemize}
	\item HW5 (2.4.2, 2.4.3, 2.5.2, 2.5.3) due tomorrow 10am
	\item HW6 due next tuesday 10am (7 is due next next double check if i need to)
	\item hw6 q1 is that amortization schedule that I made last time
	\item get these notes into Charlie Cruz's notation (or someone not me lol)
	\item tex lect 7 from kevin and the earlier lect from kat
	\item today's mantra: write math first then tex in dead moments
\end{itemize}

\section{3.2. Amortized Loans with Level Payments}
%overwrite default section numbering
\subsection{3.2.1: Remark}
$$L = OB_{0}= a_{\actuarialangle{n}i}$$
Prospective View:
$$\begin{aligned}
	OB_t &= a_{\actuarialangle{n-t}i}
\end{aligned}$$
Retroactive View:
$$\begin{aligned}
	OB_t &= L(1+i)^{t} - S_{\actuarialangle{t}i} \\
\end{aligned}$$
From the above, we can observe that,
$$\begin{aligned}
	I_t &= a_{n-(t-1)}i = a_{\actuarialangle{n-t}i}i\\
	I_t &= ia_{n-t+1}\\
	I_t &= i \left(\frac{1-v^{n-t+1}}{i}\right)\\
	I_t &= OB_{t-1}i = 1-v^{n-t+1}\\
\end{aligned}$$

\subsection{3.2.2: Remark:}
Sometimes we like to know the total interest paid. In particular, 
$$\begin{aligned}
	I &= I_1 + I_2 + \cdots + I_n\\
	  &= OB_0i + OB_1i + \cdots + OB_{n-1}i\\
	  &= \text{re-write in terms of $a_{\actuarialangle{n}i}$}\\
	  &= a_{\actuarialangle{n}i}i + a_{\actuarialangle{n-1}i}i + \cdots + a_{\actuarialangle{n-(n-1)}i}i\\
	  %factor out i
	  &= (a_{\actuarialangle{n}} + a_{\actuarialangle{n-1}} + \cdots + a_{\actuarialangle{n-(n-1)}})i\\
	  &= (\frac{1-v^{n}}{i} + \frac{1-v^{n-1}}{i} + \cdots + \frac{1-v^{1}}{i})i\\ 
	  &= (1-v^n) + (1-v^{n-1}) + \cdots + (1-v^{1})\\
	  &= n - (v^n + v^{n-1} + \cdots + v^{1})\\
	  &= n - a_{\actuarialangle{n}}
\end{aligned}$$
\subsection{3.2.3: Example hw6 pr2 see GB:}
Sam borrows $X$ for 10 years at an annual effective rate of 8\%. 
If he pays the principal and the accumulated interest in one lump
sum at the end of 10 years, he would pay 468.05 more in interest
than if he repaid the loan with ten level payments at the end of each year. 
Calculate $X$.

\subsection{3.2.4: Example hw6 pr3 see GB:}
A loan L is to be repaid with 40 payments of 100 at the end of each quarter. Interest
on the loan is charged at a nominal rate $i$, where $0<i<1$, convertible monthly.
The outstanding principlals immediately after the 8th and 24th payments
are 2308.15 and 1345.50, respectively. Calculate the amount of interest
repaid in the 15th payment.

\subsection{3.2.5: Example hw6 pr4 see GB:}
A loan at a nominal rate of 24\% convertible monthly is to be repaid with equal payments at the
end of each month for $2n$ months. The nth payment consists of equal amounts of interest and
principal. Calculate $n$.



 
% --------------------------------------------------------------
%     You don't have to mess with anything below this line.
% --------------------------------------------------------------
 
\end{document}

	% % --------------------------------------------------------------
% This is all preamble stuff that you don't have to worry about.
% Head down to where it says "Start here"
% --------------------------------------------------------------
 
\documentclass[12pt]{article}
\usepackage{actuarialsymbol}
 
\usepackage[margin=1in]{geometry} 
\usepackage{amsmath,amsthm,amssymb}

 
\newcommand{\N}{\mathbb{N}}
\newcommand{\Z}{\mathbb{Z}}
 
\newenvironment{theorem}[2][Theorem]{\begin{trivlist}
\item[\hskip \labelsep {\bfseries #1}\hskip \labelsep {\bfseries #2.}]}{\end{trivlist}}
\newenvironment{lemma}[2][Lemma]{\begin{trivlist}
\item[\hskip \labelsep {\bfseries #1}\hskip \labelsep {\bfseries #2.}]}{\end{trivlist}}
\newenvironment{exercise}[2][Exercise]{\begin{trivlist}
\item[\hskip \labelsep {\bfseries #1}\hskip \labelsep {\bfseries #2.}]}{\end{trivlist}}
\newenvironment{problem}[2][Problem]{\begin{trivlist}
\item[\hskip \labelsep {\bfseries #1}\hskip \labelsep {\bfseries #2.}]}{\end{trivlist}}
\newenvironment{question}[2][Question]{\begin{trivlist}
\item[\hskip \labelsep {\bfseries #1}\hskip \labelsep {\bfseries #2.}]}{\end{trivlist}}
\newenvironment{corollary}[2][Corollary]{\begin{trivlist}
\item[\hskip \labelsep {\bfseries #1}\hskip \labelsep {\bfseries #2.}]}{\end{trivlist}}
 
\begin{document}
 
% --------------------------------------------------------------
%                         Start here
% --------------------------------------------------------------
 
\title{Lecture 10}%replace X with the appropriate number
\author{Clark Saben\\ %replace with your name
TOI} %if necessary, replace with your course title
 
\maketitle

\section{Todos}

\begin{itemize}
	\item HW5 we got 100\% (you have the tex and pdf)
	\item HW6 due next tuesday 10am (7 is due next next double check if i need to)
	\item hw6 q1 is that amortization schedule that I made last time
	\item get these notes into Charlie Cruz's notation (or someone not me lol)
	\item tex lect 7 from kevin and the earlier lect from kat
	\item today's mantra: write math first then tex in dead moments
	\item Final is take home (due wednesday 4/26)
	\item schedule a tex all lecture notes day
\end{itemize}

\section{3.3 The Sinking Fund Method of Loan Repayment}
Recall the amortization method
$$
\begin{aligned}
	K_{t} &= I_t + PR_t \\
\end{aligned}
$$
\subsection{3.3.1 Remark:}
Consider the following payment method:
\begin{enumerate}
	\item Pay the interest at time t, $I_t$
	\item Invest the principal, $PR_t$ into a fund (sinking fund) at a arate $j$ where $j>i$
	\item Pay off $L$ using a single payment at time $t=n$
\end{enumerate}

\pagebreak

Observe that ("break even"),
$$
\begin{aligned}
	(K+L_i)S_{\actuarialangle{n}j} &= L\\
\end{aligned}
$$
in practice,
$$
\begin{aligned}
	(K+L_i)S_{\actuarialangle{n}j} &= L+extra\\
\end{aligned}
$$

\subsection{3.3.2 e.g.}
Paul borrows 10,000 for 10 years at an annual effective rate $i$. He accumulates
the amount needed to repay the loan by using a sinking fund. He makes 10 payments
of $X$ at the end of each year, each payment includes interest on the loan and the
payment into the sinking fund which earns an annual effective rate of $8\%$. If the annual
effective rate on loan had been 2$i$, his total annual payment would have been 1.5$X$. Calculate $i$.\\

Solution:\\
Observe that X=interest + principal\\
Principal going into the sinking is X-interest

consequently,
$$
\begin{aligned}
	(X-10000i)S_{\actuarialangle{10}.08}&=10000\\
	X-10000i &= 690.29
\end{aligned}


 
% --------------------------------------------------------------
%     You don't have to mess with anything below this line.
% --------------------------------------------------------------
 
\end{document}

	% % --------------------------------------------------------------
% This is all preamble stuff that you don't have to worry about.
% Head down to where it says "Start here"
% --------------------------------------------------------------
 
\documentclass[12pt]{article}
\usepackage{actuarialsymbol}
 
\usepackage[margin=1in]{geometry} 
\usepackage{amsmath,amsthm,amssymb}

 
\newcommand{\N}{\mathbb{N}}
\newcommand{\Z}{\mathbb{Z}}
 
\newenvironment{theorem}[2][Theorem]{\begin{trivlist}
\item[\hskip \labelsep {\bfseries #1}\hskip \labelsep {\bfseries #2.}]}{\end{trivlist}}
\newenvironment{lemma}[2][Lemma]{\begin{trivlist}
\item[\hskip \labelsep {\bfseries #1}\hskip \labelsep {\bfseries #2.}]}{\end{trivlist}}
\newenvironment{exercise}[2][Exercise]{\begin{trivlist}
\item[\hskip \labelsep {\bfseries #1}\hskip \labelsep {\bfseries #2.}]}{\end{trivlist}}
\newenvironment{problem}[2][Problem]{\begin{trivlist}
\item[\hskip \labelsep {\bfseries #1}\hskip \labelsep {\bfseries #2.}]}{\end{trivlist}}
\newenvironment{question}[2][Question]{\begin{trivlist}
\item[\hskip \labelsep {\bfseries #1}\hskip \labelsep {\bfseries #2.}]}{\end{trivlist}}
\newenvironment{corollary}[2][Corollary]{\begin{trivlist}
\item[\hskip \labelsep {\bfseries #1}\hskip \labelsep {\bfseries #2.}]}{\end{trivlist}}
 
\begin{document}
 
% --------------------------------------------------------------
%                         Start here
% --------------------------------------------------------------
 
\title{Lecture 11}%replace X with the appropriate number
\author{Clark Saben\\ %replace with your name
TOI} %if necessary, replace with your course title
 
\maketitle

\section{Todos}

\begin{itemize}
	\item get these notes into Charlie Cruz's notation (or someone not me lol)
	\item tex lect 7 from kevin and the earlier lect from kat
	\item today's mantra: write math first then tex in dead moments
	\item Final is take home (due wednesday 4/26). mix of open ended and multiple choice
	\item schedule a tex all lecture notes day
\end{itemize}

\section{4.2 Bonds}
\begin{enumerate}
\item face amount
\item number of coupons
\item coupon rate
\item maturity
\item coupon payment
\end{enumerate}

Price = $Fra_{\actuarialangle{n}j} + Cv_j^n$
\subsection{4.2.1}
E.g.
Calculate the price of a 5\% 10 year bond with a face amount of a 100,
semiannual coupons, and yield rate of 7.2\% compounded semiannually.
Solution:\\
$$
\begin{aligned}
	F &= 100 \\
	n &= 20 \\
	C &= 100
	r &= 0.05 \\
	j &=  \frac{7.2}{2} = 3.6\%\\
\end{aligned}
$$
Thus, GREENBOOK
$$
\begin{aligned}
	\text{Price} &= $Fra_{\actuarialangle{n}j} + Cv_j^n$
\end{aligned}

 
% --------------------------------------------------------------
%     You don't have to mess with anything below this line.
% --------------------------------------------------------------
 
\end{document}


\end{document}
