% --------------------------------------------------------------
% This is all preamble stuff that you don't have to worry about.
% Head down to where it says "Start here"
% --------------------------------------------------------------
 
\documentclass[12pt]{article}
 
\usepackage[margin=1in]{geometry} 
\usepackage{amsmath,amsthm,amssymb}
\usepackage{actuarialsymbol}
 
\newcommand{\N}{\mathbb{N}}
\newcommand{\Z}{\mathbb{Z}}
 
\newenvironment{theorem}[2][Theorem]{\begin{trivlist}
\item[\hskip \labelsep {\bfseries #1}\hskip \labelsep {\bfseries #2.}]}{\end{trivlist}}
\newenvironment{lemma}[2][Lemma]{\begin{trivlist}
\item[\hskip \labelsep {\bfseries #1}\hskip \labelsep {\bfseries #2.}]}{\end{trivlist}}
\newenvironment{exercise}[2][Exercise]{\begin{trivlist}
\item[\hskip \labelsep {\bfseries #1}\hskip \labelsep {\bfseries #2.}]}{\end{trivlist}}
\newenvironment{problem}[2][Problem]{\begin{trivlist}
\item[\hskip \labelsep {\bfseries #1}\hskip \labelsep {\bfseries #2.}]}{\end{trivlist}}
\newenvironment{question}[2][Question]{\begin{trivlist}
\item[\hskip \labelsep {\bfseries #1}\hskip \labelsep {\bfseries #2.}]}{\end{trivlist}}
\newenvironment{corollary}[2][Corollary]{\begin{trivlist}
\item[\hskip \labelsep {\bfseries #1}\hskip \labelsep {\bfseries #2.}]}{\end{trivlist}}
 
\begin{document}
 
% --------------------------------------------------------------
%                         Start here
% --------------------------------------------------------------
 
\title{Theory of Interest}%replace X with the appropriate number
\author{Clark Saben\\ %replace with your name
}% 3/6/23 lecture} %if necessary, replace with your course title
% \title{}%replace X with the appropriate number
%sample envs
% \begin{theorem}[Theorem 1.1]
% \begin{lemma}[Lemma 1.1]
% \begin{exercise}[Exercise 1.1]
% \begin{problem}[Problem 1.1]
% \begin{question}[Question 1.1]
% \begin{corollary}[Corollary 1.1]
\maketitle

\begin{section}{3/6/23 Lecture}

%iitemize
Homework and Logistics
\hline 
\begin{itemize}
	\item wednesday 4pm math meeting
	\item illustrations are in green small handbook associated with this day
	\item Get a credit card plan
	\item get a rider such that after your life insurance expires it keeps building up.
		you can have a rider to pull money out 60\% if you are terminally ill. having a 
		will makes things better.
	\item Make a master tex file for this folder
	\item fix errors and make equations look nice

\end{itemize}
\hline

\begin{question}[Example 2.2.6]
	a consider and annuity as in e.g. 2.2.4 with the following adjustments. 
	suppose that the interest rate is $12\%$ per annun for the first 10
	months with payments of $X$ each, and the rate doubles to $24\%$ for
	the rest of the term with payments of $2X$ each. Determine the level
	payment for each period.
\end{question}

\boxed{Solution:}\\
Recall that the accumulated value of this annuity (see e.g. 2.2.4) is 10000,
consequently (keep in mind we accumulate the 10 years), 
\begin{align*}
	%busted because I cannot find actuarial angle
	{XS_{\actuarialangle{10}i_{1}}\left(1+i_{2}\right)^{10} + 2XS_{\actuarialangle{10}i_{2}}} &= 10000 \\
	X  \frac{{}[\left(1.01)^{10} -1]}{0.01} + 2X \frac{{}[\left(1.02)^{10} -1]}{0.02} &= 10000 \\
	X &= 288.58
	% X\left
\end{align*}

\pagebreak

\begin{question}[Example 2.2.7]
	b Find the monthly payment for a 30 year fixed loan of 200,000 with 
	APR of 4.5\% compounded monthly, and payments made at the end of each month.
\end{question}

\boxed{Solution:}\\

This problem can be approached in two ways:\\

\begin{enumerate}
	\item  Future value of a loan (number lie from 1 to 360 where loan is given at 1).\\
	X(Accum Value) should = Accum value of the Loan\\
	$XS\actuarialangle{n}i = 200,000(1+i)$\\
\item Present value of a loan (number lie from 1 to 360 where loan is given at 360).\\ 
	PV discussion happens at time=0
\end{enumerate}

\hline

%new sign
Let $v$ be the discount factor, where $v = \frac{1}{1+i}$, and $i$ is the interest rate.\\
Then the sum of all discounts should = the PV of the loan.\\

%make this prettier
$Xv + Xv^2+ ... + Xv^{360} = 200,000$\\
$Xv\left(1+v+v^2+...+v^{359}\right) = 200,000$ \text{(notice we have the geometriic series)}\\
$Xv\frac{[1-v^{360}]}{[1-v]} = 200,000$\\

Now,  find $X$\\


2.2.8 Remark:
In general, we denote\\
$v + v^2 +..+ v^n = a\actuarialangle{n}i$, where $v=\frac{1}{1+i}$\\

2.2.9 Definition: \\
$a\actuarialangle{n}i$ is the present value of an annuity-immediate with level payments of 1 occuring 
at the end of each month for n months at a rate of $i$ where a\actuarialangle{n}i = v+ v^2 +..+ v^n$\\

\begin{theorem}[Theorem 2.2.10]
	1  a_{\actuarialangle{n}i} = v+v^2+...+v^n = \frac{1-v^{n}}{i}\\
\end{theorem}

%align this
\begin{proof}
	Observe that,\\
	$a_{\actuarialangle{n}i} &= v+v^2+...+v^n = v[1+v+v^2+...+v^{n-1}]$\\
	$ &= v\frac{[1-v^{n}]}{[1-v]}$\\

	It suffices to show that $\frac{v}{1-v} = \frac{1^}{i}$\\

	To this end, since $v = \frac{1}{1+i}$, we have\\
	$1-v = 1 - \frac{1}{1+i} = \frac{1+i}{1+i} - \frac{1}{1+i} = \frac{i}{1+i}$\\
	$\frac{v}{1-v} = \frac{1}{1+i} / \frac{i}{1+i} = \frac{1+i}{i} * \frac{1}{1+i} = \frac{1}{i}\\
	$a_{\actuarialangle{n}i} = \frac{v}{1-v^n}\left(1-v^{n}\right) = \frac{1}{i}\left(1-v^{n}\right) = \frac{1-v^n}{i}
\end{proof}

\begin{problem}[2.1.11 Example]
	a In the preceding example, determine the amount of each monthly payment if no payment
	is made for the first 12 months.
    
\end{problem}





\end{section}
% --------------------------------------------------------------
%     You don't have to mess with anything below this line.
% --------------------------------------------------------------
 
\end{document}
