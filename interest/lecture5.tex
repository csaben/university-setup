% --------------------------------------------------------------
% This is all preamble stuff that you don't have to worry about.
% Head down to where it says "Start here"
% --------------------------------------------------------------
 
\documentclass[12pt]{article}
\usepackage{actuarialsymbol}
 
\usepackage[margin=1in]{geometry} 
\usepackage{amsmath,amsthm,amssymb}

 
\newcommand{\N}{\mathbb{N}}
\newcommand{\Z}{\mathbb{Z}}
 
\newenvironment{theorem}[2][Theorem]{\begin{trivlist}
\item[\hskip \labelsep {\bfseries #1}\hskip \labelsep {\bfseries #2.}]}{\end{trivlist}}
\newenvironment{lemma}[2][Lemma]{\begin{trivlist}
\item[\hskip \labelsep {\bfseries #1}\hskip \labelsep {\bfseries #2.}]}{\end{trivlist}}
\newenvironment{exercise}[2][Exercise]{\begin{trivlist}
\item[\hskip \labelsep {\bfseries #1}\hskip \labelsep {\bfseries #2.}]}{\end{trivlist}}
\newenvironment{problem}[2][Problem]{\begin{trivlist}
\item[\hskip \labelsep {\bfseries #1}\hskip \labelsep {\bfseries #2.}]}{\end{trivlist}}
\newenvironment{question}[2][Question]{\begin{trivlist}
\item[\hskip \labelsep {\bfseries #1}\hskip \labelsep {\bfseries #2.}]}{\end{trivlist}}
\newenvironment{corollary}[2][Corollary]{\begin{trivlist}
\item[\hskip \labelsep {\bfseries #1}\hskip \labelsep {\bfseries #2.}]}{\end{trivlist}}
 
\begin{document}
 
% --------------------------------------------------------------
%                         Start here
% --------------------------------------------------------------
 
\title{Lecture 5}%replace X with the appropriate number
\author{Clark Saben\\ %replace with your name
TOI} %if necessary, replace with your course title
 
\maketitle

\section{Todos}

\begin{itemize}
	\item QUIZ Celebration of learning posted next weds DUE Tuesday 3/28 9am
	\item get these notes into Charlie Cruz's notation (or someone not me lol)
	\item HW5 due sometime after C2
\end{itemize}

%ideally working from a master.tex that shows previous days would be nice
\section{2.4.2 e.g. == HW5 q1}
Jeff deposits 100 at the end of each year for 13 years into a fund X. 
Jen deposits 100 at the end of each year for 13 years into a fund Y. 
Fund X earns an annual effective rate of 15\% for the first five years and 
annual ERI of 6\% for the remaining eight years. Fund Y earns an annual 
effective rate $i$. Both funds have the same accumulated value at the end of 13
years. What is the value of $i$?

$$
\left(100S_{\actuarialangle{5}.15}\right)\left(1+0.06\right)^8 + \left(100S_{\actuarialangle{8}.06}\right)= 100S_{\actuarialangle{13}.i}
$$

$$
i=?
$$

Remark: So far, our discussion has considered annuities for which frequency and interest conversion periods
are the same. In general, this may not be the case.

\section{2.4.3  e.g. == HW5 q2}
\begin{itemize}
	\item An annuity immediate has ten monthly payments of 1, and the quoted interest rate $i^{(4)}=8\%$. Determine its PV and its FV\\
	\item If the quoted rate is an annual effective rate of 6\%, find the PV and the fV of the annuity.
\end{itemize}

\pagebreak

\section{2.5: Increasing and Decreasing Annuities} 
\subsection{2.5.1: Increasing Annuities}
Consider an annuity in which successive payments follow an arithmetic progression, 
namely: 1, 2, 3, 4, 5, 6, 7, 8, 9, n.\\

Let the PV of this annuity be $X$. That is,
%number line
$$
X = 1 + 2 + 3 + 4 + 5 + 6 + 7 + 8 + 9 + n
$$

$PV$ == Present Value
$$
PV = \left(IA \right)_{\actuarialangle{n}} = v + 2v^2 + \cdots + nv^n
$$

Since $PV = X$, we have:

Eq1 (how to tack this to the side of the eqn and reference it later in tex?)
$$
X = v + 2v^2 + \cdots + nv^n = X
$$ 

Eq2
$$
1 + 2v + 3v^2 + \cdots + nv^{n-1} = \frac{X}{v} = X \left(1+i\right)
$$

$$
\text{Eq1 - Eq2} \implies \left(1+i \right)X-X = 1 + v + v^2 + \cdots + v^{n-1} - nv^n
$$

That is,
$$
iX = 1 + v + 2v^2 + \cdots + v^{n-1} - nv^n
$$

recall;
$$
a^{..}_{\actuarialangle{n}i} =  1 + v + 2v^2 + \cdots + v^{n-1}
$$

$$
iX \implies a^{..}_{\actuarialangle{n}i} - nv^n
$$

Therefore,
$$ 
X = \frac{a^{..}_{\actuarialangle{n}}- nv^n}{i}
$$

i.e.
$$
PV =  \left(IA \right)_{\actuarialangle{n}} = \frac{a^{..}_{\actuarialangle{n}}- nv^n}{i}
$$

Similarly (proof is left as an exercise to the reader),
$$
FV = \left(IS \right)_{\actuarialangle{n}} = \frac{s^{..}_{\actuarialangle{n}}- nv^n}{i}
$$

%underline
Consequently, the PV of an increasing perpetuity is given by,
$$
IA_{\actuarialangle{\infty}} = \frac{1}{id} = \left(\frac{1}{i} \right) \left(\frac{1+i}{i}\right) = \frac{1}{i^2} + \frac{1}{i}
$$

 
% --------------------------------------------------------------
%     You don't have to mess with anything below this line.
% --------------------------------------------------------------
 
\end{document}
