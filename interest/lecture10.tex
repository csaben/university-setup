% --------------------------------------------------------------
% This is all preamble stuff that you don't have to worry about.
% Head down to where it says "Start here"
% --------------------------------------------------------------
 
\documentclass[12pt]{article}
\usepackage{actuarialsymbol}
 
\usepackage[margin=1in]{geometry} 
\usepackage{amsmath,amsthm,amssymb}

 
\newcommand{\N}{\mathbb{N}}
\newcommand{\Z}{\mathbb{Z}}
 
\newenvironment{theorem}[2][Theorem]{\begin{trivlist}
\item[\hskip \labelsep {\bfseries #1}\hskip \labelsep {\bfseries #2.}]}{\end{trivlist}}
\newenvironment{lemma}[2][Lemma]{\begin{trivlist}
\item[\hskip \labelsep {\bfseries #1}\hskip \labelsep {\bfseries #2.}]}{\end{trivlist}}
\newenvironment{exercise}[2][Exercise]{\begin{trivlist}
\item[\hskip \labelsep {\bfseries #1}\hskip \labelsep {\bfseries #2.}]}{\end{trivlist}}
\newenvironment{problem}[2][Problem]{\begin{trivlist}
\item[\hskip \labelsep {\bfseries #1}\hskip \labelsep {\bfseries #2.}]}{\end{trivlist}}
\newenvironment{question}[2][Question]{\begin{trivlist}
\item[\hskip \labelsep {\bfseries #1}\hskip \labelsep {\bfseries #2.}]}{\end{trivlist}}
\newenvironment{corollary}[2][Corollary]{\begin{trivlist}
\item[\hskip \labelsep {\bfseries #1}\hskip \labelsep {\bfseries #2.}]}{\end{trivlist}}
 
\begin{document}
 
% --------------------------------------------------------------
%                         Start here
% --------------------------------------------------------------
 
\title{Lecture 10}%replace X with the appropriate number
\author{Clark Saben\\ %replace with your name
TOI} %if necessary, replace with your course title
 
\maketitle

\section{Todos}

\begin{itemize}
	\item HW5 we got 100\% (you have the tex and pdf)
	\item HW6 due next tuesday 10am (7 is due next next double check if i need to)
	\item hw6 q1 is that amortization schedule that I made last time
	\item get these notes into Charlie Cruz's notation (or someone not me lol)
	\item tex lect 7 from kevin and the earlier lect from kat
	\item today's mantra: write math first then tex in dead moments
	\item Final is take home (due wednesday 4/26)
	\item schedule a tex all lecture notes day
\end{itemize}

\section{3.3 The Sinking Fund Method of Loan Repayment}
Recall the amortization method
$$
\begin{aligned}
	K_{t} &= I_t + PR_t \\
\end{aligned}
$$
\subsection{3.3.1 Remark:}
Consider the following payment method:
\begin{enumerate}
	\item Pay the interest at time t, $I_t$
	\item Invest the principal, $PR_t$ into a fund (sinking fund) at a arate $j$ where $j>i$
	\item Pay off $L$ using a single payment at time $t=n$
\end{enumerate}

\pagebreak

Observe that ("break even"),
$$
\begin{aligned}
	(K+L_i)S_{\actuarialangle{n}j} &= L\\
\end{aligned}
$$
in practice,
$$
\begin{aligned}
	(K+L_i)S_{\actuarialangle{n}j} &= L+extra\\
\end{aligned}
$$

\subsection{3.3.2 e.g.}
Paul borrows 10,000 for 10 years at an annual effective rate $i$. He accumulates
the amount needed to repay the loan by using a sinking fund. He makes 10 payments
of $X$ at the end of each year, each payment includes interest on the loan and the
payment into the sinking fund which earns an annual effective rate of $8\%$. If the annual
effective rate on loan had been 2$i$, his total annual payment would have been 1.5$X$. Calculate $i$.\\

Solution:\\
Observe that X=interest + principal\\
Principal going into the sinking is X-interest

consequently,
$$
\begin{aligned}
	(X-10000i)S_{\actuarialangle{10}.08}&=10000\\
	X-10000i &= 690.29
\end{aligned}


 
% --------------------------------------------------------------
%     You don't have to mess with anything below this line.
% --------------------------------------------------------------
 
\end{document}
