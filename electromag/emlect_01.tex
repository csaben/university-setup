\documentclass{article}
\usepackage[utf8]{inputenc}
\usepackage{amsmath}
\usepackage{amsfonts} % enable the use of \mathbb
\usepackage{physics}



\title{Lecture 1}
\author{Clark Saben}

\begin{document}
% define commands for special state vectors, p means positive, n means negative
\newcommand{\pz}{\ket{\vb{+z}}}
\newcommand{\nz}{\ket{\vb{-z}}}
\newcommand{\px}{\ket{\vb{+x}}}
\newcommand{\nx}{\ket{\vb{-x}}}
\newcommand{\py}{\ket{\vb{+z}}}
\newcommand{\ny}{\ket{\vb{-y}}}

% writes expression of |+x> etc. in |z> basis
\newcommand{\pxexpr}{ \frac{1}{\sqrt{2}} \pz + \frac{1}{\sqrt{2}} \nz}
\newcommand{\nxexpr}{ \frac{1}{\sqrt{2}} \pz - \frac{1}{\sqrt{2}} \nz}
\newcommand{\pyexpr}{ \frac{1}{\sqrt{2}} \pz + \frac{i}{\sqrt{2}} \nz}
\newcommand{\nyexpr}{ \frac{1}{\sqrt{2}} \pz - \frac{i}{\sqrt{2}} \nz}
\maketitle

\section{2/21/23 lecture}

\begin{enumerate}
	\item $V=0$ everywhere (trivial)
	\item more meaningful solutions agree with the boundary conditions(b.c.) and vary in some region as $V(x,y,z)$.
\end{enumerate}
	%see notes

Dirichlet b.c.:\\
\begin{enumerate}
\item $V(x=5m)= 0 = m(5) +b  => b=-5m$
\item $V(x=1)= 4 = m(1) + (-5m) = -4m => m=-1$
\end{enumerate}

Hence,\\

$V(x) = (5-x )$ Volts \\
$E_x = -1$ V/m \\

\begin{itemize}
	\item Solution is generally a linear function
	\item consider $V(x+a) = mx + b  + ma$
	\item and      $V(x-a) = mx + b  - ma$\\
\end{itemize}

$V(x)$ is the average of these expressions (explicitly shown below).\\

$V(x) = \frac{1}{2} (V(x+a) + V(x-a))$\\

Laplace's equations is like an averageing instruction (1D, 2D, 3D).\\

%see notes for fig

$\nabla^2 V = 0$\\
$\implies$ No maxium or minimum inside of the volume of the space. 
And the only extrema exists at the boundary\\

i.e. $\left(\frac{dV}{dn}= \frac{-\sigma}{\epsilon_0}\right)$\\


There are two types of b.c.'s:\\
\begin{enumerate}
	\item Dirichlet b.c. : $V$ is fixed by some external means at the surface $dV$ (e.g. grounded)
	\item Neumann b.c. : The value of $\vec{\nabla}V . \hat{n}$ at the 
		surface AV is fixed ( the normal derivate is fixed, i.e. 
		$\frac{dV}{dn} = ____$ is fixed
\end{enumerate}

There are actually two uniqueness theorems but we will only be using the first one in this class.\\
\begin{enumerate}
	\item Theorem 1: With either b.c. (1) or (2) chosen, at surface $dV$,
		there is a unique solution $V(x,y,z)$ to Laplace's equation
		in a region of space.
\end{enumerate}

Proof:\\
Suppose we have a function $f=V_1 - V_2$, where $V_1$ and $V_2$ satisfy \nabla^2 V_i = 0.\\

%uhh purely copilot
% Then, $\nabla^2 f = \nabla^2 (V_1 - V_2) = \nabla^2 V_1 - \nabla^2 V_2 = 0 - 0 = 0$\\
\begin{align*}
{\int_V \vec{\nabla} \cdot \left(f\vec{\nabla}f\right) d^3x} &= \int_V\left(\vec{\nabla}f \cdot \vec{\nabla}f) + f \left(\nabla^2 f \right)\right)d^3x\\
							     &= \int_V \left(\vec{\nabla}f \cdot \vec{\nabla}f \right) d^3x \text{  Div. Thm}\\
							     &= \int_{dV} f\vec{\nabla}f \cdot d\vec{a}\\
\end{align*}
$\nabla^2f$ becomes zero, hence the simplification\\

A) Suppose we pick a situation where Dirichlet b.c. conditions are used.\\
Then, $V_1$ and $V_2$ are both solutions to Laplace's equation. Hence, they 
have the same value at the boundary ($dV \left(V_1=V_2=V_{fixed})$)\\

$f=0$ in volume $V$ too, or else on extrema would occur inside region.\\

Suppose $f>0$ inside $V$, in order to go to zero at $dV$ or $f$ would have some
max insde $V$ ($\nabla^2 f \neq 0$): contradiction.\\

$\implies f=0$  or $V_1=V_2$ is unique.\\

B) Suppose we pick a situation where Neumann b.c. conditions are used.
Then, $\vec{\nabla}f = 0$ at $dV$ (since $\vec{\nabla}V_1 . \hat{n}_ = \vec{\nabla}V_2 . \hat{n}_$)\\

%see notes 

%how to make this indent same for all three lines
Within Volume $V$ $\vec{\nabla}f=0$ too since,\\
\oint_{aV} f\vec{\nabla}f \cdot d\vec{a} = 0 \implies \int_V |\vec{\nabla}f|^2 d^3x$\\
since $\vec{\nabla}f = 0$ everywhere\\

Here, we have a gauge freedom. This can be seen because 
$V_1 = V_2 + V_0$ for some $V_0$ (constant).\\

\begin{itemize}
	\item Physics only cares about how the $V$ changes ( $\vec{E} = -\vec{\nabla}V$))
\end{itemize}


e.g. of B) Parallel Plates\\

Say, $|\vec{B}| = |\frac{\grad{V}}{d}| = 1000$ V/m\\

%see notes for figure

$\vec{E} $ will always be in direction of lower potential.

Separation of variables\\

Goal: Find $V(x,y,z)$ such that $\nabla^2 V = 0$\\

In 3D we have 6 total b.c.'s.\\

Working under assumption of $V(x,y,z) = f(x^{1})g(x^{2})h(x^{3})$ s.t. superscripts represent different x's\\

The solutions depend on coordinate system:
\begin{enumerate}
	\item Cartesian: x,y,z or $x^1, x^2, x^3$ $\implies$ sines/cosines, exponentials
	\item Cylindrical: s,$\rho$,z $\implies$ Bessel functions: $J^n$, $K^{n}$
	\item Spherical: r, $\theta$, $\phi$ $\implies$ Legendre Polynomials
		(i.e. new trancendental functions)
\end{enumerate}

\begin{enumerate}
	\item Cartesian
		%see notes
\end{enumerate}

\begin{itemize}
	\item $c<0$, $c<-k^2 \implies $\frac{f"}{f}=-k^{2} $ s.t. k is constant
		%notes
\end{itemize}
		%see notes

e.g. 2D) 
%see notes for image








%see notes











\end{document}
