% --------------------------------------------------------------
% This is all preamble stuff that you don't have to worry about.
% Head down to where it says "Start here"
% --------------------------------------------------------------
 
\documentclass[12pt]{article}
 
\usepackage[margin=1in]{geometry} 
\usepackage{amsmath,amsthm,amssymb}
 
\newcommand{\N}{\mathbb{N}}
\newcommand{\Z}{\mathbb{Z}}
 
\newenvironment{theorem}[2][Theorem]{\begin{trivlist}
\item[\hskip \labelsep {\bfseries #1}\hskip \labelsep {\bfseries #2.}]}{\end{trivlist}}
\newenvironment{lemma}[2][Lemma]{\begin{trivlist}
\item[\hskip \labelsep {\bfseries #1}\hskip \labelsep {\bfseries #2.}]}{\end{trivlist}}
\newenvironment{exercise}[2][Exercise]{\begin{trivlist}
\item[\hskip \labelsep {\bfseries #1}\hskip \labelsep {\bfseries #2.}]}{\end{trivlist}}
\newenvironment{problem}[2][Problem]{\begin{trivlist}
\item[\hskip \labelsep {\bfseries #1}\hskip \labelsep {\bfseries #2.}]}{\end{trivlist}}
\newenvironment{question}[2][Question]{\begin{trivlist}
\item[\hskip \labelsep {\bfseries #1}\hskip \labelsep {\bfseries #2.}]}{\end{trivlist}}
\newenvironment{corollary}[2][Corollary]{\begin{trivlist}
\item[\hskip \labelsep {\bfseries #1}\hskip \labelsep {\bfseries #2.}]}{\end{trivlist}}
 
\begin{document}
 
% --------------------------------------------------------------
%                         Start here
% --------------------------------------------------------------
 
\title{A4-Draft\_CSaben}
\author{Clark Saben}

 
\maketitle
1. (6 points) Let $\left\langle G_{1}, \cdot\right\rangle$ and 
	$\left\langle G_{2}, *\right\rangle$ be groups with identities 
	$e_{1}$ and $e_{2}$, respectively. 
	Suppose that $f: G_{1} \rightarrow G_{2}$ is an isomorphism. 
	Prove that $f\left(e_{1}\right)=e_{2}$. Write - when you 
	multiply elements of $G_{1}$, and $*$ when you multiply elements of $G_{2}$.

\begin{proof}{1} 
	Since $f: G_{1} \rightarrow G_{2}$ is an isomorphism,
	that is, it preserves the operation the for any elements 
	$a,b \in G_1, $ we have $f(a \cdot b) = f(a)*f{b}$.
	Note that by definition, $f: G_{1} \rightarrow G_{2}$ is also bijective, and so surjective.
	Let $b \in G_2$ is the image of some $a \in G_1$ since 
	$f$ is surjective. Then, $b = f(a)$ for some $a \in G_1$.
	Let $e_1 \in G_1$ and $e_2 \in G_2$ be the identities
	of each respectve group. Then,
	$$
	\begin{aligned}
		f(e_1)*f(a) &= f(e_1 \cdot a) \\
			    &= f(a) \\
			    &= b \\
	\end{aligned}
	$$
	and
	$$
	\begin{aligned}
		f(e_1)*b &= b \\
			 &= e_2 * b. 
	\end{aligned}
	$$
	Thus, $f(e_1)*b = e_2 * b$ because $b = f(a)$, and $f(e_1)*f(a) = f(e_1 \cdot a)$
	by the isomorphism property of $f$. Thus, $f(e_1) = e_2$.

\end{proof}

2. (6 points) Let $G$ be a group and let $a \in G$ be an element of order 6 . Construct a Cayley table (written in $\mathrm{LT}_{\mathrm{EX}} \mathrm{X}$ ) for the cyclic subgroup $\langle a\rangle$ of $G$ generated by $a$. Each element in your table must be written as $a^{r}$, where $r \in\{0,1, \ldots, 5\}$. If you wish, you can write $a^{0}$ as $e$, and $a^{1}$ as $a$.

$$
\begin{array}{|c|c|c|c|c|c|c|}
\hline
\cdot & e & a & a^2 & a^3 & a^4 & a^5 \\
\hline
e & e & a & a^2 & a^3 & a^4 & a^5 \\
\hline
a & a & a^2 & a^3 & a^4 & a^5 & e \\
\hline
a^2 & a^2 & a^3 & a^4 & a^5 & e & a \\
\hline
a^3 & a^3 & a^4 & a^5 & e & a & a^2 \\
\hline
a^4 & a^4 & a^5 & e & a & a^2 & a^3 \\
\hline
a^5 & a^5 & e & a & a^2 & a^3 & a^4 \\
\hline
\end{array}
$$

\newpage

3. (8 points) Let $D_{4}$ be the group of symmetries of the square. \\

(a) List all distinct cyclic subgroups of $D_{4}$. Write each cyclic subgroup as $\langle a\rangle=\{\ldots\}$, using the symbols from Class Notes for Chapter 7: $R_{0}, R_{90}, R_{180}, R_{270}, \rho_{A}, \rho_{B}, \rho_{H}$, and $\rho_{V}$. For example, $\left\langle\rho_{A}\right\rangle=\left\{R_{90}, R_{180}, \rho_{V}\right\}$ (that is incorrect). Do not repeat. \\

(b) Find a subgroup $H$ of $D_{4}$ such that $H \neq D_{4}$ and $H$ is not cyclic. Explain why your $H$ is not cyclic. \\

Hint: See Assignment 2, but be careful if your table for $D_{4}$ was not correct. \\

(a)
\begin{itemize}
    \item \( \langle R_{0} \rangle = \{R_{0}\} \)%  since $R_{0}$ is the identity.
    \item \( \langle R_{90} \rangle = \{R_{0}, R_{90}, R_{180}, R_{270}\} \)
    \item \( \langle R_{180} \rangle = \{R_{0}, R_{180}\} \)
    \item \( \langle R_{270} \rangle = \{R_{0}, R_{270}\} \)
    \item \( \langle \rho_{A} \rangle = \{R_{0}, \rho_{A}\} \)
    \item \( \langle \rho_{B} \rangle = \{R_{0}, \rho_{B}\} \)
    \item \( \langle \rho_{H} \rangle = \{R_{0}, \rho_{H}\} \)
    \item \( \langle \rho_{V} \rangle = \{R_{0}, \rho_{V}\} \)
\end{itemize}
(b)

The subgroup $H$ of $D_4$ such that $H \neq D_4$ and $H$ is not cyclic is,
$H = \{R_{0}, \rho_{H}, \rho_{V}, \rho_{A}\, \rho_{B}\}$.
Recall, from (a),
\begin{itemize}
    \item \( \langle R_{0} \rangle = \{R_{0}\} \)%  since $R_{0}$ is the identity.
    \item \( \langle \rho_{A} \rangle = \{R_{0}, \rho_{A}\} \)
    \item \( \langle \rho_{B} \rangle = \{R_{0}, \rho_{B}\} \)
    \item \( \langle \rho_{H} \rangle = \{R_{0}, \rho_{H}\} \)
    \item \( \langle \rho_{V} \rangle = \{R_{0}, \rho_{V}\} \).
\end{itemize}
Thus, no element in $H$ generates $H$ therefore this subgroup of $D_4$ is not cyclic.




\end{document}
