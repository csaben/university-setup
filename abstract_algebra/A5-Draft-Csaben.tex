% --------------------------------------------------------------
% This is all preamble stuff that you don't have to worry about.
% Head down to where it says "Start here"
% --------------------------------------------------------------
 
\documentclass[12pt]{article}
 
\usepackage[margin=1in]{geometry} 
\usepackage{amsmath,amsthm,amssymb}
 
\newcommand{\N}{\mathbb{N}}
\newcommand{\Z}{\mathbb{Z}}
 
\newenvironment{theorem}[2][Theorem]{\begin{trivlist}
\item[\hskip \labelsep {\bfseries #1}\hskip \labelsep {\bfseries #2.}]}{\end{trivlist}}
\newenvironment{lemma}[2][Lemma]{\begin{trivlist}
\item[\hskip \labelsep {\bfseries #1}\hskip \labelsep {\bfseries #2.}]}{\end{trivlist}}
\newenvironment{exercise}[2][Exercise]{\begin{trivlist}
\item[\hskip \labelsep {\bfseries #1}\hskip \labelsep {\bfseries #2.}]}{\end{trivlist}}
\newenvironment{problem}[2][Problem]{\begin{trivlist}
\item[\hskip \labelsep {\bfseries #1}\hskip \labelsep {\bfseries #2.}]}{\end{trivlist}}
\newenvironment{question}[2][Question]{\begin{trivlist}
\item[\hskip \labelsep {\bfseries #1}\hskip \labelsep {\bfseries #2.}]}{\end{trivlist}}
\newenvironment{corollary}[2][Corollary]{\begin{trivlist}
\item[\hskip \labelsep {\bfseries #1}\hskip \labelsep {\bfseries #2.}]}{\end{trivlist}}
 
\begin{document}
 
% --------------------------------------------------------------
%                         Start here
% --------------------------------------------------------------
 
\title{A5-Draft\_CSaben}
\author{Clark Saben}

 
\maketitle


1. (7 points) Consider the alternating group $A_{4}$ and its subgroup $H=\langle(134)\rangle$.

\begin{problem}{1a}
(a) Calculate the right cosets of $H$ in $A_{4}$. Do not repeat. Each right coset must be written as $H \alpha=\{\ldots\}$, for example,

$$
H(1)=\{(1),(134),(143)\}
$$

Hint: Do not use any elements that are not in $A_{4}$. For example, $H(123)$ is a coset of $H$ in $A_{4}$, but $H(12)$ is not.
\end{problem}

$$
\begin{aligned}
	H(1) &= \{(1),(134),(143)\} \\
	H(123) &= \{(123),(1234),(124)\} \\
	H(132) &= \{(132),(14)(23),(243)\} \\
	H(421) &= \{(421),(423),(42)(13)\} \\
\end{aligned}
$$

\begin{problem}{1b}
(b) Show that $H$ is not a normal subgroup of $A_{4}$.

Hint: Again, do not use any elements that are not in $A_{4}$.
\end{problem}

For $H$ to be a normal subgroup of $A_4$, we must have $gH = Hg$ for all $g \in A_4$. However, we can see that $H(123) \neq (123)H$ since $(123)H = \{(123),(234),(14)(23)\}$ and $H(123) = \{(123),(1234),(124)\}$.

\begin{problem}{1c}
(c) What is the index $\left(A_{4}: H\right)$ of $H$ in $A_{4}$ ? Justify your answer.
\end{problem}	

The index of $H$ in $A_4$ is the number of right cosets of $H$ in $A_4$. Since we have 4 cosets, the index is 4. In other words,
$\text{index}=\left(A_{4}: H\right) = 4$.

\pagebreak

2. (7 points) Let $G=\{x \in \mathbb{R}: x \neq-1\}$. Define an operation $*$ on $G$ by $a * b=a+b+a b$. For example, $2 * 3=2+3+2 \cdot 3=11$. Then $\langle G, *\rangle$ is a group. (Take this as given.) Recall that $\mathbb{R}^{*}$ is the group of nonzero real numbers with ordinary multiplication.

\begin{problem}{2a}
(a) Prove that $f: \mathbb{R}^{*} \rightarrow G$ defined by $f(x)=x-1$ is a homomorphism.
\end{problem}

\begin{proof}
	Let $a,b \in \mathbb{R}^*$. Then we have
	$$
	\begin{aligned}
		f(ab) &= ab - 1 \\
	\end{aligned} 
	$$
	and, 
	$$
	\begin{aligned}
		f(a)f(b) &= (a-1)*(b-1) \\
		&= (a-1) + (b-1) + (a-1)(b-1) \\
		&= a + b - 2 + ab - a - b + 1 \\
		&= ab - 1 \\
	\end{aligned} 
	$$
	Therefore, $f(ab) = f(a)f(b)$ and $f$ is a homomorphism.
\end{proof}

\begin{problem}{2b}
(b) Find the kernel of $f$. Justify your answer.
\end{problem}

The kernel of $f$ is the set of all elements in $\mathbb{R}^*$ that map to the identity in $G$. Since $f(x) = x - 1$, we have $f(x) = 0$ when $x = 1$. Thus, the kernel of $f$ is $\{1\}$.

\pagebreak

3. (6 points) Let $H$ be any subgroup of $G$ and let $K$ be a normal subgroup of $G$. Define a subset $S$ of $G$ by


$$
S=\{h k: h \in H \text { and } k \in K\}
$$


\begin{problem}{3}
Prove that $S$ is a subgroup of $G$.

Hint: Recall that $K$ is a normal subgroup of $G$ if and only if $K a=a K$ for every $a \in G$.
\end{problem}
\begin{proof}
	Note that $S$ is nonempty since $H$ and $K$ are nonempty. 
	Note $K$ is a normal subgroup of $G$ if and only if $Ka = aK$ for every $a \in G$.
	Let $a,b \in S$. Then $a = hk$ and $b = h'k'$ for some $h,h' \in H$ and $k,k' \in K$. 
	Then we have,
	$$
	\begin{aligned}
		ab^{-1} &= (hk)(h'k')^{-1} \\
		&= (hk)(k'^{-1}h'^{-1}) \\
		&= h(kk'^{-1})h'^{-1} \\
	\end{aligned}
	$$
	Since $H$ is a subgroup of $G$, $h(kk'^{-1})h'^{-1} \in H$.
	%why?
	%Because both H and G are subgroups of G, and the product of two elements in a group is also in the group.
	Since $K$ is a normal subgroup of $G$, $kk'^{-1} \in K$.
	%why?
	% bc Kh=hK => K=hKh^-1, so in this case K=kk'^-1
	So, $h(kk'^{-1})h'^{-1} \in H$ and $kk'^{-1} \in K$.
	% more logic:
	% Also, $h(kk'^{-1}) \in H$ and $h'^{-1} \in H$.
	% Thus, $h(kk'^{-1})h'^{-1} \in S$.
	Therefore, $S$ is a subgroup of $G$.
\end{proof}




% 1. (6 points) Let $\left\langle G_{1}, \cdot\right\rangle$ and 
% 	$\left\langle G_{2}, *\right\rangle$ be groups with identities 
% 	$e_{1}$ and $e_{2}$, respectively. 
% 	Suppose that $f: G_{1} \rightarrow G_{2}$ is an isomorphism. 
% 	Prove that $f\left(e_{1}\right)=e_{2}$. Write - when you 
% 	multiply elements of $G_{1}$, and $*$ when you multiply elements of $G_{2}$.

% \begin{proof}{1} 
% 	Since $f: G_{1} \rightarrow G_{2}$ is an isomorphism,
% 	that is, it preserves the operation the for any elements 
% 	$a,b \in G_1, $ we have $f(a \cdot b) = f(a)*f{b}$.
% 	Note that by definition, $f: G_{1} \rightarrow G_{2}$ is also bijective, and so surjective.
% 	Let $b \in G_2$ is the image of some $a \in G_1$ since 
% 	$f$ is surjective. Then, $b = f(a)$ for some $a \in G_1$.
% 	Let $e_1 \in G_1$ and $e_2 \in G_2$ be the identities
% 	of each respectve group. Then,
% 	$$
% 	\begin{aligned}
% 		f(e_1)*f(a) &= f(e_1 \cdot a) \\
% 			    &= f(a) \\
% 			    &= b \\
% 	\end{aligned}
% 	$$
% 	and
% 	$$
% 	\begin{aligned}
% 		f(e_1)*b &= b \\
% 			 &= e_2 * b. 
% 	\end{aligned}
% 	$$
% 	Thus, $f(e_1)*b = e_2 * b$ because $b = f(a)$, and $f(e_1)*f(a) = f(e_1 \cdot a)$
% 	by the isomorphism property of $f$. Thus, $f(e_1) = e_2$.

% \end{proof}

% 2. (6 points) Let $G$ be a group and let $a \in G$ be an element of order 6 . Construct a Cayley table (written in $\mathrm{LT}_{\mathrm{EX}} \mathrm{X}$ ) for the cyclic subgroup $\langle a\rangle$ of $G$ generated by $a$. Each element in your table must be written as $a^{r}$, where $r \in\{0,1, \ldots, 5\}$. If you wish, you can write $a^{0}$ as $e$, and $a^{1}$ as $a$.

% $$
% \begin{array}{|c|c|c|c|c|c|c|}
% \hline
% \cdot & e & a & a^2 & a^3 & a^4 & a^5 \\
% \hline
% e & e & a & a^2 & a^3 & a^4 & a^5 \\
% \hline
% a & a & a^2 & a^3 & a^4 & a^5 & e \\
% \hline
% a^2 & a^2 & a^3 & a^4 & a^5 & e & a \\
% \hline
% a^3 & a^3 & a^4 & a^5 & e & a & a^2 \\
% \hline
% a^4 & a^4 & a^5 & e & a & a^2 & a^3 \\
% \hline
% a^5 & a^5 & e & a & a^2 & a^3 & a^4 \\
% \hline
% \end{array}
% $$

% \newpage

% 3. (8 points) Let $D_{4}$ be the group of symmetries of the square. \\

% (a) List all distinct cyclic subgroups of $D_{4}$. Write each cyclic subgroup as $\langle a\rangle=\{\ldots\}$, using the symbols from Class Notes for Chapter 7: $R_{0}, R_{90}, R_{180}, R_{270}, \rho_{A}, \rho_{B}, \rho_{H}$, and $\rho_{V}$. For example, $\left\langle\rho_{A}\right\rangle=\left\{R_{90}, R_{180}, \rho_{V}\right\}$ (that is incorrect). Do not repeat. \\

% (b) Find a subgroup $H$ of $D_{4}$ such that $H \neq D_{4}$ and $H$ is not cyclic. Explain why your $H$ is not cyclic. \\

% Hint: See Assignment 2, but be careful if your table for $D_{4}$ was not correct. \\

% (a)
% \begin{itemize}
%     \item \( \langle R_{0} \rangle = \{R_{0}\} \)%  since $R_{0}$ is the identity.
%     \item \( \langle R_{90} \rangle = \{R_{0}, R_{90}, R_{180}, R_{270}\} \)
%     \item \( \langle R_{180} \rangle = \{R_{0}, R_{180}\} \)
%     \item \( \langle R_{270} \rangle = \{R_{0}, R_{270}\} \)
%     \item \( \langle \rho_{A} \rangle = \{R_{0}, \rho_{A}\} \)
%     \item \( \langle \rho_{B} \rangle = \{R_{0}, \rho_{B}\} \)
%     \item \( \langle \rho_{H} \rangle = \{R_{0}, \rho_{H}\} \)
%     \item \( \langle \rho_{V} \rangle = \{R_{0}, \rho_{V}\} \)
% \end{itemize}
% (b)

% The subgroup $H$ of $D_4$ such that $H \neq D_4$ and $H$ is not cyclic is,
% $H = \{R_{0}, \rho_{H}, \rho_{V}, \rho_{A}\, \rho_{B}\}$.
% Recall, from (a),
% \begin{itemize}
%     \item \( \langle R_{0} \rangle = \{R_{0}\} \)%  since $R_{0}$ is the identity.
%     \item \( \langle \rho_{A} \rangle = \{R_{0}, \rho_{A}\} \)
%     \item \( \langle \rho_{B} \rangle = \{R_{0}, \rho_{B}\} \)
%     \item \( \langle \rho_{H} \rangle = \{R_{0}, \rho_{H}\} \)
%     \item \( \langle \rho_{V} \rangle = \{R_{0}, \rho_{V}\} \).
% \end{itemize}
% Thus, no element in $H$ generates $H$ therefore this subgroup of $D_4$ is not cyclic.




\end{document}
