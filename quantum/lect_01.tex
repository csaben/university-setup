\documentclass{article}
\usepackage[utf8]{inputenc}
\usepackage{amsmath}
\usepackage{amsfonts} % enable the use of \mathbb
\usepackage{physics}



\title{Lecture 1}
\author{Clark Saben}

\begin{document}
% define commands for special state vectors, p means positive, n means negative
\newcommand{\pz}{\ket{\vb{+z}}}
\newcommand{\nz}{\ket{\vb{-z}}}
\newcommand{\px}{\ket{\vb{+x}}}
\newcommand{\nx}{\ket{\vb{-x}}}
\newcommand{\py}{\ket{\vb{+z}}}
\newcommand{\ny}{\ket{\vb{-y}}}

% writes expression of |+x> etc. in |z> basis
\newcommand{\pxexpr}{ \frac{1}{\sqrt{2}} \pz + \frac{1}{\sqrt{2}} \nz}
\newcommand{\nxexpr}{ \frac{1}{\sqrt{2}} \pz - \frac{1}{\sqrt{2}} \nz}
\newcommand{\pyexpr}{ \frac{1}{\sqrt{2}} \pz + \frac{i}{\sqrt{2}} \nz}
\newcommand{\nyexpr}{ \frac{1}{\sqrt{2}} \pz - \frac{i}{\sqrt{2}} \nz}
\maketitle

\section{2/21/23 lecture (Lakroski 4.5)}

E. Infinite Well\\

The 'Program'\\

If \hat{H} is Time Independent:
\hat{U}{t}|\Psi(t=0)\rangle = e^{-i \hat{H} t \hbar}

'Stationary' States: \hat{H}||Psi_{n}\rangle = E_{n}|\Psi_{n}\rangle\\

So,
|\Psi(o)\rangle = \sum_{n=1}^{\infty} \langle\Psi{n}|\Psi_{n}\rangle\\

\implies \hat{U}(t)|\rangle = \sum_{n}e^{-i \hat{H} t/ \hbar} 

%see notes

General Solution of the TSDE if \hat{H} is Time indpendent:

%see notes

All we need in are the |\Psi_{n}\rangle

Time independent  shrodinger equation:

%see notes 

In matrix representation (homework, quiz, exam):

\mathbb{H}\vec{v}_n = E_n \vec{v}_n
\begin{align*}
	{H_{ij}} &= \langle\vec{v}_i|\hat{H}|\vec{v}_j\rangle \text{Eigenvalue problem}
		 &= %see notes \text{Matrix elements }
\end{align*}

We do this if we don't know the eigenvectors associated with \mathbb{H} and we want
to know them.\\

In Eigenstate basis:

%diagnol matrix
%see notes

In another basis:
%see notes
Now you have stuff that are non-zero off diagnol elements.

The way you do this is you write the eigenvalues of the 2nd.

The other way to do it is in the The Wave Function Representation:

$\hat{H}(x)\Psi(x) = E_{n}\Psi_{n}(x)$

In 1D:\\

% $\left(%see notes \right) % see notes
$\left(\frac{-\hbar}{2m}\frac{d^2}{dx^2} + V(x)\right)\Psi_{n}(x) = E_{n}\Psi_{x}(x)$\\

$->$ Second ODE\\
In class: Look for symmetries and/or conserverd quantities.
(he makes a note about softwares that handle this being worth a lot)


%see notes

\implies $[\hat{H}, \hat{O}] = 0$; $d<\hat{O}>/dt = 0$\\
$->$ Is conserved.\\

Advantage: We can identify conserved quantities using classical intuition.\\

Once we have identified a conseved quantity( \hat{O}) then it has the same
eigenstates as $\hat{H}$.\\

E.g. Infinite well\\

%see notes
Q1: Which quantities are conseved in the region $x\in(0,a))$ for a particle
in a infinite potential energy well of length a (assume elastic collisions):

answer: (D) Magnitude of Momentum and Energy\\

Explanation:

Angular Momentum:\\
$\vec{L} = \vec{r} x \vec{p}$\\
%see notes

$|\hat{p}|$ and $\hat{H}$ are conserved, so:\\
$[\hat{H}, |\hat{p}|] = 0$\\
$\implies \hat{H}$ and $|\hat{p}|$ have simultaneous eigenstates.\\

Q2: What are appropiate eigenfunctions of the momentum operator?\\

answer: (B) $->$ $Ae^{-ipx/h})$

Explanation: %see notes

For $|\hat{p}|$ we can make a superposition of $\pm$ Eigenstates:\\
$\Psi(x) = \alpha e^{\frac{ipx}{\hbar}} + \beta e^{-\frac{ipx}{\hbar}}$\\'
Another way to write this (SHO);\\
$= Asin(\frac{px}{\hbar}) + Bcos(\frac{px}{\hbar})$\\

Say, $k$ = $\frac{p}{\hbar}$ \implies \Psi(x) = Asin(kx) + Bcos(kx)$\\

$\Psi(x)$ represents Stationary States!\\
Eigenstates of $\hat{H}$ , because $[\hat{H}, \hat{p}] = 0$\\

(language note: state and wave fn == eigenstate)\\

For a  time independent problem such as this you need the boundary conditions (i.c. is for time dependent problems).\\)

For $x>a$ or $x<0$ the wave function is zero.\\

% see notes
These are our boundary conditions. (see diagram in notes)\\
\implies$\Psi(x) = 0$\\
\implie$\Psi(a) = 0$\\


Q3 TLDR: To satisfy the boundary condition  $\Psi(0) = 0$, the constant $B$ must be zero.\\
Explanation: plug in $x=0$ into $\Psi(x) = Asin(kx) + Bcos(kx)$\\

Moral: The wave function must be zero at the boundary.\\

Q4: To satisfy the boundary condition(b.c.) $\Psi(a)=0$, the constant $k$ in the wavefunction of a 
particle is an infinite well must be; (assume $B=0$ from prev. b.c.)\\

answer: (A) $k = \frac{n\pi}{a}$ s.t. $n\in \mathbb{Z}$ but is not zero\\

$a$ and/or $n$ cannot simply be zero because otherwise the wave-fn is zero everywhere)\\.

%fraction broke lol
$k_n$ = $\frac{n\pi}{a} = $\frac{pn}{\hbar}$\\

The boundary condition quantizes momentum.\\










\end{document}
