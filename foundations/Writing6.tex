% --------------------------------------------------------------
% This is all preamble stuff that you don't have to worry about.
% Head down to where it says "Start here"
% --------------------------------------------------------------
 
\documentclass[12pt]{article}
 
\usepackage[margin=1in]{geometry} 
\usepackage{amsmath,amsthm,amssymb}
 
\newcommand{\N}{\mathbb{N}}
\newcommand{\Z}{\mathbb{Z}}
 
\newenvironment{theorem}[2][Theorem]{\begin{trivlist}
\item[\hskip \labelsep {\bfseries #1}\hskip \labelsep {\bfseries #2.}]}{\end{trivlist}}
\newenvironment{lemma}[2][Lemma]{\begin{trivlist}
\item[\hskip \labelsep {\bfseries #1}\hskip \labelsep {\bfseries #2.}]}{\end{trivlist}}
\newenvironment{exercise}[2][Exercise]{\begin{trivlist}
\item[\hskip \labelsep {\bfseries #1}\hskip \labelsep {\bfseries #2.}]}{\end{trivlist}}
\newenvironment{problem}[2][Problem]{\begin{trivlist}
\item[\hskip \labelsep {\bfseries #1}\hskip \labelsep {\bfseries #2.}]}{\end{trivlist}}
\newenvironment{question}[2][Question]{\begin{trivlist}
\item[\hskip \labelsep {\bfseries #1}\hskip \labelsep {\bfseries #2.}]}{\end{trivlist}}
\newenvironment{corollary}[2][Corollary]{\begin{trivlist}
\item[\hskip \labelsep {\bfseries #1}\hskip \labelsep {\bfseries #2.}]}{\end{trivlist}}
 
\begin{document}
 
% --------------------------------------------------------------
%                         Start here
% --------------------------------------------------------------
 
\title{Writing Assignment 6}%replace X with the appropriate number
\author{Clark Saben\\ %replace with your name
Foundations of Mathematics} %if necessary, replace with your course title
 
\maketitle

%prove with contradiction
 
\begin{theorem}{WA 6.1}
	If $n \in \mathbb{N}$, then $1+3+5+...+(2n-1) = n^2$. 
\end{theorem}
\begin{proof}
Let $n \in \mathbb{N}$. We will proceed by induction.\\
Firstly, when $n=1$,
\begin{align*}
	(2(1)-1) &= (1)^2\\
		 &= 1^2\\
		 &= 1
\end{align*}
Next, to begin our inductive step, let us assume that for some $k$, such that $k \in \mathbb{N}$, $1+3+5+...+(2k-1) = k^2$. We can
show that when $n=k+1$,

\begin{align*}
	1+3+5+...+(2k-1) + (2k+2-1) &= (k+1)^2\\
				    &= k^2 + 2k + 1\\
\end{align*}
Finally, by adding $2k+2-1$ to both sides of $1+3+5+...+(2k-1) = k^2$, we get
\begin{align*}
	{1+3+5+...+(2k-1) + (2k+2-1)} &= k^2 + (2k+2-1)\\
				      &= k^2 + (2k+2-1)\\
				      &= k^2 + 2k + 1\\
\end{align*}
Therefore, by induction if $n \in \mathbb{N}$, then $1+3+5+...+(2n-1) = n^2$.
\end{proof}
 
\pagebreak

\begin{theorem}{WA 6.2}
	For every $n \in \mathbb{N}$, $2^n+1 \le 3^{n}$.\\
\end{theorem}
\begin{proof}
	Let $n \in \mathbb{N}$. We will proceed by induction.\\
	Firstly, when $n=1$,
	\begin{align*}
		2^1+1 &= 3^1\\
		    3 &= 3^1\\
		    3 &= 3 
\end{align*}
Next, to begin our inductive step, let us assume that for some $k$, such that $k \in \mathbb{N}$, $2^k+1 \le 3^k$. We can then
show that when $n=k+1$,
\begin{align*}
	2^{k+1}+1 &= 3^{k+1}\\
		  &= (3)3^k\\
		  &= (3)(2^k+1) \text{ (given our assumption above)}\\
		  &= (2+1)(2^k+1)\\
		  &= 2^{k+1}+2+2k+1\\
		  &= 2^{k+1}+2k+3\\
\end{align*}
Clearly, $2^{k+1}+1 \le 2^{k+1}+2k+3$, so we can conclude by induction that for every $n \in \mathbb{N}$, $2^n+1 \le 2^{n+1}$.


\end{proof}
 

    



 
% --------------------------------------------------------------
%     You don't have to mess with anything below this line.
% --------------------------------------------------------------
 
\end{document}
