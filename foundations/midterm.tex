% --------------------------------------------------------------
% This is all preamble stuff that you don't have to worry about.
% Head down to where it says "Start here"
% --------------------------------------------------------------
 
\documentclass[12pt]{article}
 
\usepackage[margin=1in]{geometry} 
\usepackage{amsmath,amsthm,amssymb}
 
\newcommand{\N}{\mathbb{N}}
\newcommand{\Z}{\mathbb{Z}}
 
\newenvironment{theorem}[2][Theorem]{\begin{trivlist}
\item[\hskip \labelsep {\bfseries #1}\hskip \labelsep {\bfseries #2.}]}{\end{trivlist}}
\newenvironment{lemma}[2][Lemma]{\begin{trivlist}
\item[\hskip \labelsep {\bfseries #1}\hskip \labelsep {\bfseries #2.}]}{\end{trivlist}}
\newenvironment{exercise}[2][Exercise]{\begin{trivlist}
\item[\hskip \labelsep {\bfseries #1}\hskip \labelsep {\bfseries #2.}]}{\end{trivlist}}
\newenvironment{problem}[2][Problem]{\begin{trivlist}
\item[\hskip \labelsep {\bfseries #1}\hskip \labelsep {\bfseries #2.}]}{\end{trivlist}}
\newenvironment{question}[2][Question]{\begin{trivlist}
\item[\hskip \labelsep {\bfseries #1}\hskip \labelsep {\bfseries #2.}]}{\end{trivlist}}
\newenvironment{corollary}[2][Corollary]{\begin{trivlist}
\item[\hskip \labelsep {\bfseries #1}\hskip \labelsep {\bfseries #2.}]}{\end{trivlist}}
 
\begin{document}
 
% --------------------------------------------------------------
%                         Start here
% --------------------------------------------------------------
 
\title{Foundations Midterm}%replace X with the appropriate number
\author{Clark Saben\\ %replace with your name
Foundations of Mathematics} %if necessary, replace with your course title
 
\maketitle

%prove with contradiction
 
\begin{theorem}{1}
	Assume $a,b \in \mathbb{Z}$. If $a^2 \left(b^2 -2b \right)$ is odd, then $a$ and $b$ 
	is odd.
\end{theorem}
\begin{proof}
	Let $a$ and $b$ be odd and $a^2 \left(b^2 -2b \right)$ be odd. By definition 2.1
	if $a$ and $b$ are even then it can be shown,
	\begin{align*}
		{a^2 \left(b^2 -2b \right)} &= 2k^2 [\left(2j\right)^2 -2 \left(2j \right)^2] \\
					    &= \left(4k^2\right)[4j^2 - 4j]\\
					    &= 2  \left(2k^2\right) \left(2j^2 - 2j \right), \\
	\end{align*}
	which is an even number. This is a  false statement. Therefore if $a^2 \left(b^2 -2b \right)$ is odd then $a$ and $b$ must also
	be odd.
\end{proof}

%horizontal line
\hrule

\begin{theorem}{2}
	Given an integer $a$, if $7|4a$ then $7|a$.
\end{theorem}
\begin{proof}
	Let a be an integer and $7|4a$. By definition 2.1, since $7\nmid4$ then 7 must
	divide $a$ for $7|4a$ to be true. Therefore, if $7|4a$ then $7|a$.
\end{proof}

%horizontal line
\hrule

\begin{theorem}{3}
	If $n\in \mathbb{Z}$, then $5n^2 + 3n +7$ is odd.
\end{theorem}
\begin{proof}
	Let $n \in \mathbb{Z}$ and $5n^2 + 3n +7$ is odd. It can then be shown that,\\
	\begin{align}
		{5n^2 + 3n + 7 }&= 2k \text{ for some integer k}\\
		\frac{5n^2}{2} + \frac{3n}{2} + \frac{7}{2}&= k\\
	\end{align}

	This is false statement because $\frac{7}{2}$ doesn't allow 
	$\frac{5n^2}{2} + \frac{3n}{2} + \frac{7}{2}$ to be an integer
	for any $n$. Therefore, if $n \in \mathbb{Z}$ then $5n^2 + 3n +7$ is odd.
\end{proof}

\pagebreak

\begin{theorem}{5}
	Assume $a \in \mathbb{Z}$. Then $a^2 \mid a$ if and only if $a \in\{-1,0,1\}$
	%we will use proof by cases
\end{theorem}
\begin{proof}
	Let $a \in \mathbb{Z}$. Using proof by cases we will show that $a^2 \mid a$ if and only if $a \in\{-1,0,1\}$.\\

	Firstly, we will show if $a^2 \mid a$ then $a \in\{-1,0,1\}$. Consider the case $a$ is non-zero and greater than 1. 
	Then $a^2 \mid a$ implies $\frac{a}{a^2} \in \mathbb{Z}$ which is false. Similarly, when $a$ is non-zero and less than -1, 
	then $a^2 \mid a$ implies $\frac{a}{a^2} \in \mathbb{Z}$ which is false. Hence, $a$ must be either -1, 0, or 1 for $a^2 \mid a$ to be true.\\

	Secondly, we will show if $a \in\{-1,0,1\}$ then $a^2 \mid a$. We can prove this is true for each element directly by using the definition of divisibility for
	some integer k,
	%show three equations that are true, -1^2 =1k, 0^2 = 0k, 1^2 = 1k
	\begin{align*}
		\text{Case 1: } a &= -1\\
		{-1}^2 = 1k \\
		\text{Case 2: } a &= 0\\
		{0}^2 = 0k\\
		\text{Case 3: } a &= 1\\
		{1}^2 = 1k\\
	\end{align*}
	All three cases are true, therefore $a^2 \mid a$ if and only if $a \in\{-1,0,1\}$.

\end{proof}
\begin{theorem}{6}
If $A, B, C, D$ are sets and $C \subseteq A$ and $D \subseteq B$, then $D \backslash A \subseteq B \backslash C$.
\end{theorem}
\begin{proof}
Let $A, B, C, D$ be sets and $C \subseteq A$ and $D \subseteq B$. 
Let $x \in D \backslash A$. Then $x \in D$ and by definition 3.5, $x \in B$. Furthermore, because
$C \subseteq A$ it follows that $x \in B \backslash C$. Therefore $D \backslash A \subseteq B \backslash C$.

% By definition 3.5
\begin{theorem}{7}
If we define the sets $A=\{12 a+4 b: a, b \in \mathbb{Z}\}$ and $B=\{4 c: c \in \mathbb{Z}\}$, then $A \subseteq B$.
\end{theorem}
\begin{proof}
Let $A=\{12 a+4 b: a, b \in \mathbb{Z}\}$ and $B=\{4 c: c \in \mathbb{Z}\}$. Let $x \in A$ such that
$x=12 a+4 b$. It can be shown that,
\begin{align*}
x & = 12 a+4 b\\
& = 4 (3 a+ b)\\
& = 4 k \text{ for some integer k}\\
\end{align*}
Hence, $4 k \in B$. Therefore, if $A=\{12 a+4 b: a, b \in \mathbb{Z}\}$ and $B=\{4 c: c \in \mathbb{Z}\}$, then $A \subseteq B$.

	
\end{proof}

\pagebreak

\begin{theorem}{8}
Theorem 8. Given integers $a, b$, and $c$, if $a^2 \mid b$ and $b^3 \mid c$, then $a^6 \mid c$.
\end{theorem}
\begin{proof}:\\
8. Let $a, b, c \in \mathbb{Z}$\\
3. such that $a^2 \mid b$ and $b^3 \mid c$.\\
11. By the definition of divisibility, this implies that there exists integers $k$ and $\ell$ such that,\\
6. $b=a^2 k$ and $c=b^3 \ell$.\\
10. Now, substituting $b^3$, we see that,\\
2.
$$
\begin{aligned}
c & =b^3 \ell \\
& =a^6 k^3 \ell \\
& =a^6 \mathrm{~m},
\end{aligned}
$$
5. where $m=k^3 \ell$ is an integer.\\
1. Cubing both sides of $b=a^2 k$ we obtain,\\
4. $b^3=a^6 k^3$.\\
7. Hence $c=a^6 m$,\\
9. and $a^6 \mid c$ by the definition of divisibility.\\
    
\end{proof}

%horizontal line
\hrule
% --------------------------------------------------------------
%     You don't have to mess with anything below this line.
% --------------------------------------------------------------
 
\end{document}
