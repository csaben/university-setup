% --------------------------------------------------------------
% This is all preamble stuff that you don't have to worry about.
% Head down to where it says "Start here"
% --------------------------------------------------------------
 
\documentclass[12pt]{article}
 
\usepackage[margin=1in]{geometry} 
\usepackage{amsmath,amsthm,amssymb}
 
\newcommand{\N}{\mathbb{N}}
\newcommand{\Z}{\mathbb{Z}}
 
\newenvironment{theorem}[2][Theorem]{\begin{trivlist}
\item[\hskip \labelsep {\bfseries #1}\hskip \labelsep {\bfseries #2.}]}{\end{trivlist}}
\newenvironment{lemma}[2][Lemma]{\begin{trivlist}
\item[\hskip \labelsep {\bfseries #1}\hskip \labelsep {\bfseries #2.}]}{\end{trivlist}}
\newenvironment{exercise}[2][Exercise]{\begin{trivlist}
\item[\hskip \labelsep {\bfseries #1}\hskip \labelsep {\bfseries #2.}]}{\end{trivlist}}
\newenvironment{problem}[2][Problem]{\begin{trivlist}
\item[\hskip \labelsep {\bfseries #1}\hskip \labelsep {\bfseries #2.}]}{\end{trivlist}}
\newenvironment{question}[2][Question]{\begin{trivlist}
\item[\hskip \labelsep {\bfseries #1}\hskip \labelsep {\bfseries #2.}]}{\end{trivlist}}
\newenvironment{corollary}[2][Corollary]{\begin{trivlist}
\item[\hskip \labelsep {\bfseries #1}\hskip \labelsep {\bfseries #2.}]}{\end{trivlist}}
 
\begin{document}
 
% --------------------------------------------------------------
%                         Start here
% --------------------------------------------------------------
 
\title{Writing Assignment 5}%replace X with the appropriate number
\author{Clark Saben\\ %replace with your name
Foundations of Mathematics} %if necessary, replace with your course title
 
\maketitle

%prove with contradiction
 
\begin{theorem}{3.10}
	Suppose that $A, B$ and $C$ are sets. If $A \subseteq B$ and $B \subseteq C$ then $A \subseteq C$.
\end{theorem}
\begin{proof}
	Let $A, B,$ and $C$ be sets. Let $x \in A$. Since $A \subseteq B$ and $x \in A$, $x \in B$. Similarly,
	$B \subseteq C$ so $x \in C$. Therefore if $\left(A \subseteq B \right) \cap  \left(B \subseteq C \right)$ then $A \subseteq C$.
\end{proof}

%horizontal line
\hrule

\begin{theorem}{3.21b}
	If $A$ and $B$ are sets, then $\left(A \cap B \right)^{c} = A^{c} \cup B^{c}$.
\end{theorem}
\begin{proof}
	Let $A$ and $B$ be sets. To show  $\left(A \cap B \right)^{c} = A^{c} \cup B^{c}$,
	we must show that $\left(A \cap B \right)^{c} \subseteq A^{c} \cup B^{c}$ and $A^{c} \cup B^{c} \subseteq \left(A \cap B \right)^{c}$.
	Firstly, let $x \in (A \cap B)^c$, then  $x \notin A$ and $x \notin B$. Therefore, by 
	De-Morgan's law, $x \notin \left(A \cup B \right)$. If $x$ is not a  member of 
	$A$ or $B$ by definition 3.14 it follows that $x \in A^c \cup B^c$.
	Secondly, let $x \in A^c \cup B^c$. Therefore, $x \notin\left( A \cup B \right)$. By 
	De-Morgan's law, it follows that $x \notin (A \cap B)$. Hence, by definition 3.14,
	$x \in \left( A \cap B \right)^c$. Therefore, $\left(A \cap B \right)^{c} = A^{c} \cup B^{c}$.
\end{proof}


    



 
% --------------------------------------------------------------
%     You don't have to mess with anything below this line.
% --------------------------------------------------------------
 
\end{document}
