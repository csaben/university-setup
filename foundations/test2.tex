% --------------------------------------------------------------
% This is all preamble stuff that you don't have to worry about.
% Head down to where it says "Start here"
% --------------------------------------------------------------
 
\documentclass[12pt]{article}
 
\usepackage[margin=1in]{geometry} 
\usepackage{amsmath,amsthm,amssymb}
 
\newcommand{\N}{\mathbb{N}}
\newcommand{\Z}{\mathbb{Z}}
 
\newenvironment{theorem}[2][Theorem]{\begin{trivlist}
\item[\hskip \labelsep {\bfseries #1}\hskip \labelsep {\bfseries #2.}]}{\end{trivlist}}
\newenvironment{lemma}[2][Lemma]{\begin{trivlist}
\item[\hskip \labelsep {\bfseries #1}\hskip \labelsep {\bfseries #2.}]}{\end{trivlist}}
\newenvironment{exercise}[2][Exercise]{\begin{trivlist}
\item[\hskip \labelsep {\bfseries #1}\hskip \labelsep {\bfseries #2.}]}{\end{trivlist}}
\newenvironment{problem}[2][Problem]{\begin{trivlist}
\item[\hskip \labelsep {\bfseries #1}\hskip \labelsep {\bfseries #2.}]}{\end{trivlist}}
\newenvironment{question}[2][Question]{\begin{trivlist}
\item[\hskip \labelsep {\bfseries #1}\hskip \labelsep {\bfseries #2.}]}{\end{trivlist}}
\newenvironment{corollary}[2][Corollary]{\begin{trivlist}
\item[\hskip \labelsep {\bfseries #1}\hskip \labelsep {\bfseries #2.}]}{\end{trivlist}}
 
\begin{document}
 
% --------------------------------------------------------------
%                         Start here
% --------------------------------------------------------------
 
\title{Foundations Test 2}%replace X with the appropriate number
\author{Clark Saben\\ %replace with your name
Foundations of Mathematics} %if necessary, replace with your course title
 
\maketitle

\section{}
\begin{theorem}1
	If $n \in \mathbb{Z}$ and $n \geq 0$, then $\sum_{i=0}^{n}i\cdot i! = \left(n+1 \right)!-1$.
\end{theorem}
\begin{theorem}2
	The inequality $2^n \leq 2^{n+1}-2^{n-1}-1$ holds for all $n \in \mathbb{N}$.
\end{theorem}
\begin{theorem}3
	Define the relation $R$ on $\mathbb{Z}$ such that $xRy$ if and only if $3x-5y$ is even. Then $R$ is an equivalence relation.
\end{theorem}
\section{}
\begin{theorem}4
	Let $a$,$b \in \mathbb{Z}$ and $n \in \mathbb{N}$. If $a \equiv_n b$ then $a^2 \equiv_n b^2$. (recall the definition of $\equiv_n$ from Definition 7.79)
\end{theorem}
\begin{theorem}5
	For an $n \geq 4$, one can obtain $n$ dollars using only \$2 and \$5 bills.
\end{theorem}
\begin{theorem}{6 Complete but gross}
	Define $\Psi = \{(a,b): a,b \in \mathbb{Z}, b\neq 0\}$ and define a relation $\sim$ on $\Psi$ via $(a,b) \sim (c,d)$ if and only if
	$ad=bc$. Then $\sim$ is an equivalence relation.\\

	Let $\Psi = \{(a,b): a,b \in \mathbb{Z}, b\neq 0\}$ and define a relation $\sim$ on $\Psi$ via $(a,b) \sim (c,d)$ if and only if
	$ad=bc$. To show that $\sim$ is an equivalence relation, we must show that $\sim$ is reflexive, symmetric, and transitive.\\

	To show that $\sim$ is reflexive, we must show that for all $(a,b) \in \Psi$, $(a,b) \sim (a,b)$.\\
	It can be shown that, 
	\begin{align*}
		(a,b) \sim (a,b) &\Leftrightarrow ab=ba\\
				 &\Leftrightarrow ab=ab \text{ (commutative property)}\\
	\end{align*}
	Therefore, $\sim$ is reflexive.\\

	To show that $\sim$ is symmetric, we must show that for all $(a,b) \in \Psi$ and $(c,d) \in \Psi$, if $(a,b) \sim (c,d)$ then $(c,d) \sim (a,b)$.\\
	It can be shown that,
	\begin{align*}
		(a,b) \sim (c,d) &\Leftrightarrow ad=bc\\
	\end{align*}
	and that,
	\begin{align*}
		(c,d) \sim (a,b) &\Leftrightarrow cb=da\\
				 &\Leftrightarrow ad=bc \text{ (commutative property)}\\
	\end{align*}
	Therefore, $\sim$ is symmetric.\\

	To show that, $\sim$ is transitive, we must show that for all $(a,b) \in \Psi$, $(c,d) \in \Psi$, and $(e,f) \in \Psi$, if $(a,b) \sim (c,d)$ and $(c,d) \sim (e,f)$ then $(a,b) \sim (e,f)$.\\
	It can be shown that,
	\begin{align*}
		ad&=bc\\
		fad&=fbc\\
		f&=\frac{de}{c}\\
		adf&=bc(\frac{de}{c})\\
		adf&=bde\\
		af&=be\\
	\end{align*}
	Therefore, $\sim$ is transitive.\\
\end{theorem}
		
\section{(complete)}
\begin{problem}1
	Consider the following relation:\\
	$R = \{(a,a),(a,b),(a,d),(b,d),(c,c),(d,b),(d,c)\}$
\end{problem}
\begin{problem}1.1
 What elements must be add to $R$ to make it reflexive?
\end{problem}
To be reflexive, $(b,b)$ and $(d,d)$ must be added to $R$.
\begin{problem}1.1
 What elements must be add to $R$ to make it symmetric?
\end{problem}
To be symmetric, $(b,a)$, $(d,a)$ and $(c,d)$ must be added to $R$.



% --------------------------------------------------------------
%     You don't have to mess with anything below this line.
% --------------------------------------------------------------
 
\end{document}
