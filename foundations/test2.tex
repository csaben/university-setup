% --------------------------------------------------------------
% This is all preamble stuff that you don't have to worry about.
% Head down to where it says "Start here"
% --------------------------------------------------------------
 
\documentclass[12pt]{article}
 
\usepackage[margin=1in]{geometry} 
\usepackage{amsmath,amsthm,amssymb}
 
\newcommand{\N}{\mathbb{N}}
\newcommand{\Z}{\mathbb{Z}}
 
\newenvironment{theorem}[2][Theorem]{\begin{trivlist}
\item[\hskip \labelsep {\bfseries #1}\hskip \labelsep {\bfseries #2.}]}{\end{trivlist}}
\newenvironment{lemma}[2][Lemma]{\begin{trivlist}
\item[\hskip \labelsep {\bfseries #1}\hskip \labelsep {\bfseries #2.}]}{\end{trivlist}}
\newenvironment{exercise}[2][Exercise]{\begin{trivlist}
\item[\hskip \labelsep {\bfseries #1}\hskip \labelsep {\bfseries #2.}]}{\end{trivlist}}
\newenvironment{problem}[2][Problem]{\begin{trivlist}
\item[\hskip \labelsep {\bfseries #1}\hskip \labelsep {\bfseries #2.}]}{\end{trivlist}}
\newenvironment{question}[2][Question]{\begin{trivlist}
\item[\hskip \labelsep {\bfseries #1}\hskip \labelsep {\bfseries #2.}]}{\end{trivlist}}
\newenvironment{corollary}[2][Corollary]{\begin{trivlist}
\item[\hskip \labelsep {\bfseries #1}\hskip \labelsep {\bfseries #2.}]}{\end{trivlist}}
 
\begin{document}
 
% --------------------------------------------------------------
%                         Start here
% --------------------------------------------------------------
 
\title{Foundations Test 2}%replace X with the appropriate number
\author{Clark Saben\\ %replace with your name
Foundations of Mathematics} %if necessary, replace with your course title
 
\maketitle

\section{}
\begin{theorem}1
	If $n \in \mathbb{Z}$ and $n \geq 0$, then $\sum_{i=0}^{n}i\cdot i! = \left(n+1 \right)!-1$.
\end{theorem}
\begin{proof}
    Let $n \in \mathbb{Z}$ and $n \geq 0$. We will show that $\sum_{i=0}^{n}i\cdot i! = \left(n+1 \right)!-1$ via induction. For our base case, $n=0$,
    \begin{align*}
        \sum_{i=0}^{0}0\cdot 0! &= (0+1)!-1\\
	&= 0
    \end{align*}
    which is true, so we proceed with our inductive step. Let us assume that for some $k \in \mathbb{Z}$ 
    that $k \geq 0$ and $\sum_{i=0}^{k}i\cdot i! = \left(k+1 \right)!-1$. It can be shown that by adding
    $(k+1)(k+1)!)$ to both sides of the equation we get,
    \begin{align*}
	\sum_{i=0}^{k}i\cdot i! + (k+1)(k+1)! &= \left(k+1 \right)!-1 + (k+1)(k+1)!\\
    \end{align*}
    Next, by re-arranging with algebra (namely, the distributive property), we can see,
    \begin{align*}
		\sum_{i=0}^{k}i\cdot i! + (k+1)(k+1)! &= (1+(k+1))(k+1)!-1\\
						      &= (k+2)(k+1)!-1
\end{align*}
Now, notice that when we have $k=k+1$, we have,
\begin{align*}
	\sum_{i=0}^{k+1}i\cdot i! &= ((k+1)+1)!-1\\
	&= (k+2)(k+1)!-1
\end{align*}
which matches our result from the inductive step, so we have shown that\\ $\sum_{i=0}^{n}i\cdot i! = \left(n+1 \right)!-1$ for all $n \in \mathbb{Z}$ and $n \geq 0$
by process of induction.
\end{proof}
\begin{theorem}{2}
	The inequality $2^n \leq 2^{n+1}-2^{n-1}-1$ holds for all $n \in \mathbb{N}$.
	\begin{proof}
	Let $n \in \mathbb{N}$. We will show that 
	the inequality $2^n \leq 2^{n+1}-2^{n-1}-1$ holds for all $n$
	via induction. For our base case, $n=1$,
	\begin{align*}
		2^{1} &\leq 2^{1+1} -2^{1-1}-1\\
		2 &\leq 4-1-1\\
		2 &\leq 2
	\end{align*}
	which is true, so we proceed with our inductive step. Let us assume that for some $k \in \mathbb{N}$
	the inequality $2^k \leq 2^{k+1}-2^{k-1}-1$ holds for all $k$. Since $2^k$ is going to be less
	than $2^{k+1}-2^{k-1}-1$ it will also be less than that plus an additional $2^{k-1}+1$. Hence we can correctly
	re-write our inequality from,
	\begin{align*}
		2^k &\leq 2^{k+1}-2^{k-1}-1
	\end{align*}
	to,
	\begin{align*}
		2^k &\leq 2^{k+1}-2^{k-1}-1 + (2^{k-1}+1)\\
		2^k &\leq 2^{k+1}\\
		    &\leq (2)2^k\\
	\end{align*}
	Clearly, for any natural number, $k$, $2^k \leq (2)2^k$. Therefore,
	the inequality $2^n \leq 2^{n+1}-2^{n-1}-1$ holds for all $n \in \mathbb{N}$.

	\end{proof}
\end{theorem}
\begin{theorem}{3}
	Define the relation $R$ on $\mathbb{Z}$ such that $xRy$ if and only if $3x-5y$ is even. Then $R$ is an equivalence relation.\\
\end{theorem}

	\begin{proof}
	Let $x$ and $y$ $\in \mathbb{Z}$ and define relation $R$ on $\mathbb{Z}$ such that $xRy$ if and only if $3x-5y$ is even. We will show $R$ is an equivalence relation.\\

	To show $R$ is reflexive, let $x \in \mathbb{Z}$. Then $xRx$ if and only if $3x-5x$ is even. By Definition 2.1 it can be seen that,
	\begin{align*}
		3x-5x &= 2(-k) \text{ (for some $-k \in \mathbb{Z}$)}\\
		-2x &= -2k\\
	\end{align*}
	which is true for all $x \in \mathbb{Z}$, so $R$ is reflexive.\\

	To show $R$ is symmetric, let $x$ and $y$ $\in \mathbb{Z}$. Then $xRy$ if and only if $3x-5y$ is even. By Definition 2.1 it can be seen that,
	\begin{align*}
		3x-5y &= 2(-k) \text{ (for some $-k \in \mathbb{Z}$)}\\
	\end{align*}
	We can add 2x-2y to both sides to get,
	\begin{align*}
		3x-5y + 2x-2y &= 2(-k) + 2x-2y\\
		5x-3y &= 2(-k) + 2x-2y\\
		      &= -2(k-x+y)\\
	\end{align*}
	We can then multiply each side by (-1) to get,
	\begin{align*}
		(-1)5x-3y &= (-1)(-2(k-x+y))\\
		-5x+3y&= 2(k-x+y)\\
		3y-5x &= 2(k-x+y)\\
	\end{align*}
	where ($k-x+y$) is some integer. Therefore, since $3y-5x$ is even (Definition 2.1), $yRx$, so
	$R$ is symmetric.\\

	To show $R$ is transitive, let $x$,$y$,$z$ $\in \mathbb{Z}$ and $xRy$ and $yRz$. 
	Then $3x-5y$ is even and $3y-5z$ is even. By Definition 2.1 it can be seen that,
	\begin{align*}
		3x-5y &= 2(k) \text{ (for some $k \in \mathbb{Z}$)}\\
		3y-5z &= 2(l) \text{ (for some $l \in \mathbb{Z}$)}\\
	\end{align*}
	We can add the two equations to get,
	\begin{align*}
		3x-5y + 3y-5z &= 2(k) + 2(l)\\
		3x-5z - 2y &= 2(k+l)\\
			   &= 2(k+l+y)\\
	\end{align*}
	We see that $3x-5z$ is even, so $xRz$ and $R$ is transitive.\\

	In conclusion, $R$ is an equivalence relation since $R$ is reflexive, symmetric, and transitive.
\end{proof}
	
\section{}
\begin{theorem}{4}
	Let $a$,$b \in \mathbb{Z}$ and $n \in \mathbb{N}$. If $a \equiv_n b$ then $a^2 \equiv_n b^2$. (recall the definition of $\equiv_n$ from Definition 7.79)
	\end{theorem}

	\begin{proof}
	Let $a$,$b \in \mathbb{Z}$ and $n \in \mathbb{N}$. We will show that if $a \equiv_n b$ then $a^2 \equiv_n b^2$.
	By definition 7.79, $a \equiv_n b$ if and only if $a-b$ is divisible by $n$. It can thus be shown that,
	\begin{align*}
		a-b=nk \text{ (for some $k \in \mathbb{Z}$)}\\
	\end{align*}
	we can then multiply both sides by $a+b$ to get,
	\begin{align*}
		a^2+ab-ba+b^2&=(a+b)nk\\
		a^2-b^2&=(a+b)nk\\
		a^2-b^2&=np \text{  }\boxed{\text{such that $p=(a+b)k$}}\\
	\end{align*}
	Therefore, $a^2-b^2$ is divisible by $n$. 
	Thus, $a^2 \equiv_n b^2$ if $a-b$ is divisible by $n$.
\end{proof}

\begin{theorem}5
	For an $n \geq 4$, one can obtain $n$ dollars using only \$2 and \$5 bills.
\end{theorem}
\begin{proof}
	Let $n \geq 4$. We will show that one can obtain $n$ dollars using only \$2 and \$5 bills.\\
	In other words, there exist non-negative integers $a$ and $b$ such that $n = 2a + 5b$ for all $n \geq 4$.\\
	We will prove this via strong induction.\\

	For our base case we will prove the cases for $n=2$, $4$, and $5$ directly.\\
	For $n = 2$, we can use one \$2 bill: $2 = 2\cdot 1 + 5\cdot 0$ $(a=1, b=0)$\\
	For $n = 4$, we can use two \$2 bills: $4 = 2\cdot 2 + 5\cdot 0$ $(a=2, b=0)$\\
	For $n = 5$, we can use one \$5 bill: $5 = 2\cdot 0 + 5\cdot 1$ $(a=0, b=1)$\\

	For our inductive hypothesis, we will assume that for $5\leq k \leq m$ can be 
	obtained using only \$2 and \$5 bills such that $k,m \in \mathbb{Z}$.\\

	We will now prove that $k+1$ can be obtained using only \$2 and \$5 bills.
	Since $k-1$ can be obtained using only \$2 and \$5 bills, we can 
	see that,
	\begin{align*}
		k-1 = 2a + 5b \text{ (for some $a,b \in \mathbb{Z}$)}
	\end{align*}
	which can be rewritten as,
	\begin{align*}
		(k+1)-2 = 2a + 5b \\
	\end{align*}
	Hence, by adding adding a \$2 bill to the \$2 and \$5 bills used to obtain $k-1$, we can obtain $k+1$ using only \$2 and \$5 bills.\\

	Thus, we have shown via strong induction that for any $n \geq 4$, one can obtain $n$ dollars using only \$2 and \$5 bills.


	% Now, let's consider the case for $n \geq 8$. We know that any integer greater than or equal to 8 can be written as a multiple of 
	% 4 plus a remainder in the range [0, 3]. In other words, we can write $n = 4k + r$,
	% where $k$ and $r \in \mathbb{Z}$ such that $0 \leq r \leq 3$.\\

	% Notice that for the cases $r = 0, 1, 2,$ or $3$, we already have a way to obtain the amount with only \$2 and \$5 bills as demonstrated above. Therefore, for any $n \geq 8$, we can represent the value as the sum of multiples of 4 (using only \$2 bills) and one of the four remainder cases.\\

% We will now go through each of the four remainder cases.\\
% For $r=0$, we have $n = 4k \Rightarrow n = 2\cdot(2k) + 5\cdot 0$.\\
% For $r=1$, we have $n = 4k + 1 \Rightarrow n = 2\cdot(2k) + 5\cdot 1$.\\
% For $r=2$, we have $n = 4k + 2 \Rightarrow n = 2\cdot(2k) + 5\cdot() k$.\\
% For $r=3$, we have $n = 4k + 3 \Rightarrow n = 2\cdot(2k + 1) + 5\cdot 2$\\
\end{proof}
\begin{theorem}{6} 
	Define $\Psi = \{(a,b): a,b \in \mathbb{Z}, b\neq 0\}$ and define a relation $\sim$ on $\Psi$ via $(a,b) \sim (c,d)$ if and only if
	$ad=bc$. Then $\sim$ is an equivalence relation.\\
\end{theorem}

	\begin{proof}
	Let $\Psi = \{(a,b): a,b \in \mathbb{Z}, b\neq 0\}$ and define a relation $\sim$ on $\Psi$ via $(a,b) \sim (c,d)$ if and only if
	$ad=bc$. To show that $\sim$ is an equivalence relation, we must show that $\sim$ is reflexive, symmetric, and transitive.\\

	To show that $\sim$ is reflexive, we must show that for all $(a,b) \in \Psi$, $(a,b) \sim (a,b)$.
	It can be shown that, 
	\begin{align*}
		(a,b) \sim (a,b) &\Leftrightarrow ab=ba\\
				 &\Leftrightarrow ab=ab \text{ (commutative property)}\\
	\end{align*}
	Therefore, $\sim$ is reflexive.\\

	To show that $\sim$ is symmetric, we must show that for all $(a,b) \in \Psi$ and $(c,d) \in \Psi$, if $(a,b) \sim (c,d)$ then $(c,d) \sim (a,b)$.
	It can be shown that,
	\begin{align*}
		(a,b) \sim (c,d) &\Leftrightarrow ad=bc\\
	\end{align*}
	and that,
	\begin{align*}
		(c,d) \sim (a,b) &\Leftrightarrow cb=da\\
				 &\Leftrightarrow da=cb\\
				 &\Leftrightarrow ad=bc \text{ (commutative property)}\\
	\end{align*}
	Therefore, $\sim$ is symmetric.\\

	To show that, $\sim$ is transitive, we must show that for all $(a,b) \in \Psi$, $(c,d) \in \Psi$, and $(e,f) \in \Psi$, if $(a,b) \sim (c,d)$ and $(c,d) \sim (e,f)$ then $(a,b) \sim (e,f)$.
	Let $(a,b) \sim (c,d)$ and $(c,d) \sim (e,f)$.
	Since, $(a,b)\sim(c,d)$ it can be shown that,
	\begin{align*}
		ad&=bc\\
	\end{align*}
	and if we multiply each side by $f$, we get,
	\begin{align*}
		fad&=fbc\\
		adf&=bcf\\
	\end{align*}
	Also, since $(c,d)\sim(e,f)$, $cf=de$ we can re-write $f$ as $\frac{de}{c}$. We can finally show that,
	\begin{align*}
		adf&=bc(\frac{de}{c})\\
		adf&=bde\\
		af&=be \text{,}\\
	\end{align*}
	which would follow from $(a,b)\sim(e,f)$. Therefore, $\sim$ is transitive.\\

	Therefore, $\sim$ is an equivalence relation.\\
\end{proof}
		
\section{}
\begin{problem}1
	Consider the following relation:\\
	$R = \{(a,a),(a,b),(a,d),(b,d),(c,c),(d,b),(d,c)\}$
\end{problem}
\begin{problem}1.1
 What elements must be add to $R$ to make it reflexive?
\end{problem}
To be reflexive, $(b,b)$ and $(d,d)$ must be added to $R$.
\begin{problem}1.1
 What elements must be add to $R$ to make it symmetric?
\end{problem}
To be symmetric, $(b,a)$, $(d,a)$ and $(c,d)$ must be added to $R$.



% --------------------------------------------------------------
%     You don't have to mess with anything below this line.
% --------------------------------------------------------------
 
\end{document}
