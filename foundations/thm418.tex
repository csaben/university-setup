% --------------------------------------------------------------
% This is all preamble stuff that you don't have to worry about.
% Head down to where it says "Start here"
% --------------------------------------------------------------
 
\documentclass[12pt]{article}
 
\usepackage[margin=1in]{geometry} 
\usepackage{amsmath,amsthm,amssymb}
 
\newcommand{\N}{\mathbb{N}}
\newcommand{\Z}{\mathbb{Z}}
 
\newenvironment{theorem}[2][Theorem]{\begin{trivlist}
\item[\hskip \labelsep {\bfseries #1}\hskip \labelsep {\bfseries #2.}]}{\end{trivlist}}
\newenvironment{lemma}[2][Lemma]{\begin{trivlist}
\item[\hskip \labelsep {\bfseries #1}\hskip \labelsep {\bfseries #2.}]}{\end{trivlist}}
\newenvironment{exercise}[2][Exercise]{\begin{trivlist}
\item[\hskip \labelsep {\bfseries #1}\hskip \labelsep {\bfseries #2.}]}{\end{trivlist}}
\newenvironment{problem}[2][Problem]{\begin{trivlist}
\item[\hskip \labelsep {\bfseries #1}\hskip \labelsep {\bfseries #2.}]}{\end{trivlist}}
\newenvironment{question}[2][Question]{\begin{trivlist}
\item[\hskip \labelsep {\bfseries #1}\hskip \labelsep {\bfseries #2.}]}{\end{trivlist}}
\newenvironment{corollary}[2][Corollary]{\begin{trivlist}
\item[\hskip \labelsep {\bfseries #1}\hskip \labelsep {\bfseries #2.}]}{\end{trivlist}}
 
\begin{document}
 
% --------------------------------------------------------------
%                         Start here
% --------------------------------------------------------------
 
\title{Theorem 4.18}%replace X with the appropriate number
\author{Clark Saben\\ %replace with your name
Foundations of Mathematics} %if necessary, replace with your course title
 
\maketitle

%prove with contradiction
 
\begin{theorem}{4.18}
	For all integers $n \geq 3,2 \cdot 3+3 \cdot 4+\cdots+(n-1) \cdot n=\frac{(n-2)\left(n^2+2 n+3\right)}{3}$
\end{theorem}
 
\begin{proof}
	Let $n \geq 3$. We will prove that $2 \cdot 3+3 \cdot 4+\cdots+(n-1) \cdot n=\frac{(n-2)\left(n^2+2 n+3\right)}{3}$ by utilizing induction.\\

	Base case: $n=3$\\
	\begin{align*}
		{\left(3-1\right)3} &= \frac{\left(3-2 \right) \left(3^2 +2(3)+3\right))}{3}\\
			6 &= 6
	\end{align*}

	Next, for some $n=k$ such that $k \in \mathbb{N}$ and $k \geq 3$, we assume that $2 \cdot 3+3 \cdot 4+\cdots+(k-1) \cdot k=\frac{(k-2)\left(k^2+2 k+3\right)}{3}$\\
	It can be shown that for $n = k+1$ that if we add zero we get the following,
	\begin{align*}
		{2 \cdot 3+3 \cdot 4+\cdots+(k-1) \cdot k + \left((k)(k+1) - (k)(k+1) \right)} &= \left(k-2 \right) \frac{\left(k^2+2k+3 \right)}{3}\\
			2 \cdot 3+3 \cdot 4+\cdots+(k-1) \cdot k + (k)(k+1)              &= (k)(k+1) \left( \left(k-2 \right) \frac{\left(k^2+2k+3 \right)}{3} \right)\\
											 &= k^2 + k + \frac{k^3-k-6}{3}\\
											 &= \frac{3k^2 +2k-6}{3}
	\end{align*}

	We can show by substituting $k+1$ for $k$ in our original equation, that for $k \geq 3,2 \cdot 3+3 \cdot 4+\cdots+(k-1) \cdot k +(k)(k+1) =\frac{(k-2)\left(k^2+2 k+3\right)}{3}$, that 
	the right hand side can be simplified to match our preceeding equation. This can be shown as follows,
	\begin{align*}
		\frac{(k-2)\left(k^2+2 k+3\right)}{3} &= \frac{3k^2 +2k-6}{3}
	\end{align*}

	Thus, we can conclude that by induction that $2 \cdot 3+3 \cdot 4+\cdots+(n-1) \cdot n=\frac{(n-2)\left(n^2+2 n+3\right)}{3}$.



\end{proof}

    



 
% --------------------------------------------------------------
%     You don't have to mess with anything below this line.
% --------------------------------------------------------------
 
\end{document}
