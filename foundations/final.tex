% --------------------------------------------------------------
% This is all preamble stuff that you don't have to worry about.
% Head down to where it says "Start here"
% --------------------------------------------------------------
 
\documentclass[12pt]{article}
 
\usepackage[margin=1in]{geometry} 
\usepackage{amsmath,amsthm,amssymb}
 
\newcommand{\N}{\mathbb{N}}
\newcommand{\Z}{\mathbb{Z}}
 
\newenvironment{theorem}[2][Theorem]{\begin{trivlist}
\item[\hskip \labelsep {\bfseries #1}\hskip \labelsep {\bfseries #2.}]}{\end{trivlist}}
\newenvironment{lemma}[2][Lemma]{\begin{trivlist}
\item[\hskip \labelsep {\bfseries #1}\hskip \labelsep {\bfseries #2.}]}{\end{trivlist}}
\newenvironment{exercise}[2][Exercise]{\begin{trivlist}
\item[\hskip \labelsep {\bfseries #1}\hskip \labelsep {\bfseries #2.}]}{\end{trivlist}}
\newenvironment{problem}[2][Problem]{\begin{trivlist}
\item[\hskip \labelsep {\bfseries #1}\hskip \labelsep {\bfseries #2.}]}{\end{trivlist}}
\newenvironment{question}[2][Question]{\begin{trivlist}
\item[\hskip \labelsep {\bfseries #1}\hskip \labelsep {\bfseries #2.}]}{\end{trivlist}}
\newenvironment{corollary}[2][Corollary]{\begin{trivlist}
\item[\hskip \labelsep {\bfseries #1}\hskip \labelsep {\bfseries #2.}]}{\end{trivlist}}
 
\begin{document}
 
% --------------------------------------------------------------
%                         Start here
% --------------------------------------------------------------
 
\title{Foundations Test 2}%replace X with the appropriate number
\author{Clark Saben\\ %replace with your name
Foundations of Mathematics} %if necessary, replace with your course title
 
\maketitle

\section{}
\begin{theorem}1
	Given $a \in \mathbb{Z}$, if $5\mid2a$ then $5\mid a$
\end{theorem}
\begin{proof}
	Let $a \in \mathbb{Z}$ such that $5\mid2a$. This can be written as
	$2a = 5k$ for some $k \in \mathbb{Z}$. Now we consider,
	\begin{align*}
		a &=6a-4a\\
		  &=3(2a)-2(2a)\\
		  &=3(5k)-2(5k) \text{ Using hypothesis}\\
		  &=5(3k-2k)\\
	\end{align*}
	Since $3k-2k \in \mathbb{Z}$, this implies that $5\mid a$.

	% Let $a \in \mathbb{Z}$ and $5\mid2a$. Suppose for the sake of
	% contradiction $5\nmid a$. This would imply that $5\mid2$ which
	% is a contradiction. Therefore, $5\mid a$ must be true.
\end{proof}
\begin{theorem}{3}
	For all $n \in \mathbb{N}$,$ 1\cdot2 + 2\cdot3 + \cdots + n(n+1) = \frac{n(n+1)(n+2)}{3}$
\end{theorem}
	\begin{proof}
		Let $n \in \mathbb{N}$. We proceed by induction.\\
		\textbf{Base Case:} $n = 1$\\
		\begin{align*}
			1\cdot2 &= \frac{1(1+1)(1+2)}{3}\\
			1\cdot2 &= \frac{1(1+1)(1+2)}{3}\\
			2 &= \frac{1(2)(3)}{3}\\
			2 &= \frac{6}{3}\\
			2 &= 2
		\end{align*}
		\textbf{Inductive Hypothesis:} Assume that $1\cdot2 + 2\cdot3 + \cdots + k(k+1) = \frac{k(k+1)(k+2)}{3}$ for
		some integer $k$. 
		\textbf{Inductive Step:} We must show that $1\cdot2 + 2\cdot3 + \cdots + k(k+1) + (k+1)(k+2) = \frac{(k+1)(k+2)(k+3)}{3}$.
		This can be simplified on the right hand side such that it can be restated as,
		\begin{align*}
		    1\cdot2 + 2\cdot3 + \cdots + k(k+1) + (k+1)(k+2) &= \frac{k(k+1)(k+2)}{3}\\
							&= \frac{(k^2+2k+k+2)(k+3)}{3}\\
							&= \frac{k^3+3k^2+2k^2+6k+k^2+3k+2k+6}{3}\\
							&= \frac{k^3+6k^2+11k+6}{3}
		\end{align*}
		We begin by adding $(k+1)(k+2)$ to both sides of the equation in the inductive hypothesis to get
		$1\cdot2 + 2\cdot3 + \cdots + k(k+1) + (k+1)(k+2) = \frac{k(k+1)(k+2)}{3} + (k+1)(k+2)$.
		It can be then shown that,
		\begin{align*}
		    1\cdot2 + 2\cdot3 + \cdots + k(k+1) + (k+1)(k+2) &= \frac{k(k+1)(k+2)}{3} + (k+1)(k+2)\\
								     &= \frac{k(k+1)(k+2)+3(k+1)(k+2)}{3}\\
								     &= \frac{(k^2+k)(k+2)+3(k^2+3k+2)}{3}\\
								     &= \frac{k^3+2k^2+k^2+2k+3k^2+9k+6}{3}\\
								     &= \frac{k^3+3k^2+2k+3k^2+9k+6}{3}\\
								     &= \frac{k^3+6k^2+11k+6}{3}\\
		\end{align*}
		This is the same as the equation we got in the inductive step.
		Therefore, by process of induction, for all $n \in \mathbb{N}$, $1\cdot2 + 2\cdot3 + \cdots + n(n+1) = \frac{n(n+1)(n+2)}{3}$.
	\end{proof}
\begin{theorem}{4}
	Let $a,b,c \in \mathbb{Z}$ and $n\in \mathbb{N}$. If $a\equiv_nb$ then 
	$ac\equiv_nbc$
\end{theorem}
	\begin{proof}
		Let $a,b,c \in \mathbb{Z}$ and $n\in \mathbb{N}$. Let 
		$a\equiv_nb$. Then by definition 7.79 there exists an integer $k$ such that $a-b=nk$.
		It can be shown that by multiplying each side of the equation by $c$ that $ac-bc=nkc$.
		Since $kc$ is an integer, $ac\equiv_nbc$ by definition 7.79.
\end{proof}
\section{}
	
\begin{theorem}{5}
	For all integers $n \geq 0, 24\mid(5^{2n}-1)$.
\end{theorem}
\begin{proof}
	Let $n \in \mathbb{Z}$ such that $n \geq 0$. We proceed by induction.\\
	\textbf{Base Case:} $n = 0$\\
	\begin{align*}
		5^{2(0)}-1 &= 24k\\
		5^{0}-1 &= 24k\\
		1-1 &= 24k\\
		0 &= 24k
	\end{align*}
	which is true for all $k \in \mathbb{Z}$.\\
	\textbf{Inductive Hypothesis:} Assume that $24\mid(5^{2k}-1)$ for some integer $k$.
	This can be restated as $5^{2k}-1=24k$ for some integer $k$.\\
	\textbf{Inductive Step:} We must show that $24\mid(5^{2(k+1)}-1)$.
	First, by multiplying each side of the inductive hypothesis by $5^{2}$ we get,
	\begin{align*}
		5^{2}\cdot(5^{2k}-1) &= 24k\cdot5^{2}\\
		5^{2k+2}-5^{2} &= 24k\cdot5^{2}
	\end{align*}
	Secondly, we can then replace $5^{2}$ with $25$ on the left hand side to yield,
	\begin{align*}
		5^{2k+2}-25  &= 24k\cdot5^2\\
	\end{align*}
	We then make the -25 into -1-24 which yields,
	\begin{align*}
		5^{2k+2}-24-1  &= 24k\cdot5^2\\
	\end{align*}
	Next, we add over the $-24$ from the left hand side to yield,
	\begin{align*}
		5^{2k+2}-1 &= 24k\cdot5^2+24\\
	\end{align*}
	Finally, factoring out the $24$ from the right hand side of the equation yields,
	\begin{align*}
		5^{2k+2}-1 &= 24(k5^2+1)
	\end{align*}
	Since $k5^2+1$ is an integer, we have shown that $24\mid(5^{2(k+1)}-1)$.
	Therefore, by process of induction, for all integers $n \geq 0, 24\mid(5^{2n}-1)$.
\end{proof}

\begin{theorem}{6}
	For all integers $x$, if $x^2-6x+5$ is even, then $x$ is odd.
\end{theorem}
\begin{proof}
	Let $x$ be an integer such that $x^2-6x+5$ is even. We proceed by contraposition.\\
	\textbf{Contrapositive:} If $x$ is even, then $x^2-6x+5$ is odd.\\
	Let $x$ be an even integer. Then by definition 2.1 there exists an integer $k$ such
	that $x=2k$.
	We can then substitute $2k$ for $x$ in $x^2-6x+5$ to yield,
	\begin{align*}
		(2k)^2-6(2k)+5 &= 4k^2-12k+5\\
		&= 2(2k^2-6k+2)+1
	\end{align*}
	By definition 2.1, $2(2k^2-6k+2)$ is an even integer. Therefore, $x^2-6x+5$ is odd when
	$x$ is even.
	Therefore, by contraposition, for all integers $x$, if $x^2-6x+5$ is even, then $x$ is odd.
\end{proof}
\section{}
\begin{theorem}{7} 
	Let $a,b \in \mathbb{Z}$ and $n \in \mathbb{N}$. If $12a \not\equiv_n 12b$ then $a \nmid b$.
\end{theorem}
	\begin{proof}
		Let $a,b \in \mathbb{Z}$ and $n \in \mathbb{N}$. For sake of contradiction,
		suppose $12a \not\equiv_n 12b$ and $n\mid12$. By definition 2.1, there exists an integer $k$ such that
		$12=nk$. By multiplying each side by $(a-b)$ we get,
		\begin{align*}
			12(a-b) &= nk(a-b)\\
			12a-12b &= n(ka-kb)\\
		\end{align*}
		which is a contradiction because this would mean by definition 7.79 that
		$12a \equiv_n 12b$. Therefore, if $12a \not\equiv_n 12b$ then $a \nmid b$.
\end{proof}
\begin{theorem}{8} 
	For all $a,b \in \mathbb{Z},$ if $a$ is even and $b$ is odd, then 6 does not 
	divide $a^2+b^2$.
\end{theorem}
	\begin{proof}
		Let $a,b \in \mathbb{Z}$ such that by definition 2.1, there exists an integers $k,l$ such that $a=2k$ and $b=2l+1$.
		For the sake of contradiction, suppose that $6\mid(a^2+b^2)$. It can be shown that, 
		\begin{align*}
			a^2+b^2&=(2k)^2+(2l+1)^2\\
			       &=4k^2+4l^2+4l+1\\
				      &=2(2k^2+2l^2+2l)+1\\
		\end{align*}
		where $2k^2+2l^2+2l$ is an integer. 
		Therefore, by definition 2.1, $a^2+b^2$ is odd. 
		This would imply that there exists an integer $m$ such that,
		$(a^2+b^2)=6m$, which can be rewritten as,
		\begin{align*}
			\frac{(a^2+b^2)}{6} = m\\ 
		\end{align*}
		Since $a^2+b^2$ is odd, this implies $m$ is not an integer, which is a contradiction.
		% an odd number cannot be divisible by an even number (such as 6).
		Therefore, for all $a,b \in \mathbb{Z},$ if $a$ is even and $b$ is odd, then 6 does not
		divide $a^2+b^2$.
		

\end{proof}
		
\section{}
\begin{theorem}{10}
	%\cup and 
	If $A$,$B$ are sets, then $A \cap (B \setminus A) = \emptyset$
\end{theorem}
\begin{proof}

6. Let $A, B$ be sets.\\
12. Suppose, for the sake of contradiction, \\
3. that $A \cap (B \setminus A) \neq \emptyset$.\\
5. Hence, there exists an $x \in A \cap (B \setminus A)$.\\
9. By the definition of intersection,\\
1. $x \in A$ and $x \in B \setminus A$.\\
7. By the definition of set difference, \\
11. $x \in B$ and $x \notin A$.\\
8. All together, this implies that $x \in A$ \\
2. and $x \notin A$, \\
10. which is a contradiction.\\
4. Therefore, $A \cap (B \setminus A) = \emptyset$.\\

\end{proof}


% --------------------------------------------------------------
%     You don't have to mess with anything below this line.
% --------------------------------------------------------------
 
\end{document}
