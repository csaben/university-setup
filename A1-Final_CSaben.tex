% --------------------------------------------------------------
% This is all preamble stuff that you don't have to worry about.
% Head down to where it says "Start here"
% --------------------------------------------------------------
 
\documentclass[12pt]{article}
 
\usepackage[margin=1in]{geometry} 
\usepackage{amsmath,amsthm,amssymb}
 
\newcommand{\N}{\mathbb{N}}
\newcommand{\Z}{\mathbb{Z}}
 
\newenvironment{theorem}[2][Theorem]{\begin{trivlist}
\item[\hskip \labelsep {\bfseries #1}\hskip \labelsep {\bfseries #2.}]}{\end{trivlist}}
\newenvironment{lemma}[2][Lemma]{\begin{trivlist}
\item[\hskip \labelsep {\bfseries #1}\hskip \labelsep {\bfseries #2.}]}{\end{trivlist}}
\newenvironment{exercise}[2][Exercise]{\begin{trivlist}
\item[\hskip \labelsep {\bfseries #1}\hskip \labelsep {\bfseries #2.}]}{\end{trivlist}}
\newenvironment{problem}[2][Problem]{\begin{trivlist}
\item[\hskip \labelsep {\bfseries #1}\hskip \labelsep {\bfseries #2.}]}{\end{trivlist}}
\newenvironment{question}[2][Question]{\begin{trivlist}
\item[\hskip \labelsep {\bfseries #1}\hskip \labelsep {\bfseries #2.}]}{\end{trivlist}}
\newenvironment{corollary}[2][Corollary]{\begin{trivlist}
\item[\hskip \labelsep {\bfseries #1}\hskip \labelsep {\bfseries #2.}]}{\end{trivlist}}
 
\begin{document}
 
% --------------------------------------------------------------
%                         Start here
% --------------------------------------------------------------
 
\title{A1-Final\_CSaben}
\author{Clark Saben}

 
\maketitle

\begin{problem}{1}

Determine which of the following operations are associative. In each case, either write a proof or give a counter-example.

(a) the operation $*$ on $\mathbb{Z}$ defined by $a * b=a-b$;

Consider the following counter-example: $a=1, b=2, c=3$. Then,

$$
\begin{aligned}
	(a * b) * c &= (1-2) * 3 \\
		    &= -1 * 3 \\
		    &= -1-3 \\
		    &= -4 \\
\end{aligned}
$$
and,
$$
\begin{aligned}
	% a * (b * c) &= 1 * (2-3) = 1 * -1 = 1-(-1) = 1+1 = 2.
	a * (b * c) &= 1 * (2-3) \\
		    &= 1 * -1 \\
		    &= 1-(-1) \\
		    &= 1+1 \\
		    &= 2.
\end{aligned}
$$
Therefore, $(a * b) * c \neq a * (b * c)$ and the operation $*$ is not associative.
%new line
\break

(b) the operation $*$ on $\mathbb{R}$ defined by $a * b=a+2 b+a b$;

Consider the following counter-example: $a=1, b=2, c=3$. Then,
$$
\begin{aligned}
	(a * b) * c &= (1+2(2)+1(2)) * 3\\
		    &= (1+4+2) * 3 \\
		    &= 7 * 3 \\
		    &= 7+2(3)+7(3) \\
		    &= 7+6+21  \\
		    &= 34 \\
\end{aligned}
$$
and,
$$
\begin{aligned}
	a * (b * c) &= 1 * (2+2(3)+1(2)) \\
		    &= 1 * (2+6+2) \\
		    &= 1 * 10 \\
		    &= 1+2(10)+1(10) \\
		    &= 1+20+10 \\
		    &= 31.
\end{aligned}
$$
Therefore, $(a * b) * c \neq a * (b * c)$ and the operation $*$ is not associative.
\break

(c) the operation $*$ on $\mathbb{Q}^{*}=\mathbb{Q}-\{0\}$ defined by $a * b=\frac{a}{b}$;

	Consider the following counter-example, $a=1, b=2, c=3$. Then,
	% $(a * b) * c = (\frac{1}{2}) * 3 = \frac{\frac{1}{2}}{3} = \frac{1}{6}$ and
	$$
	\begin{aligned}
		(a * b) * c &= (\frac{1}{2}) * 3 \\
			    &= \frac{\frac{1}{2}}{3} \\
			    &= \frac{1}{6} \\
	\end{aligned}
	$$
	and, 
	$$
	\begin{aligned}
		a * (b * c) &= 1 * (\frac{2}{3}) \\
			    &= \frac{1}{\frac{2}{3}} \\
			    &= \frac{3}{2} \\
	\end{aligned}
	$$
	Therefore, $(a * b) * c \neq a * (b * c)$ and the operation $*$ is not associative.
\break

(d) the operation $*$ on $\mathbb{Z}$ defined by $a * b=a+b-2$.
\begin{proof}
	Let $a,b,c \in \mathbb{Z}$. Then,
	$$
	\begin{aligned}
	% $(a * b) * c = (a+b-2) * c = (a+b-2)+c-2 = a+b+c-4$ and
	(a * b) * c &= (a+b-2) * c \\
		    &= (a+b-2)+c-2 \\
		    &= a+b+c-4 \\
	\end{aligned}
	$$
	and,
	$$
	\begin{aligned}
	% $a * (b * c) = a * (b+c-2) = a+(b+c-2)-2 = a+b+c-4$. Since $a+b+c-4 = a+b+c-4$, the operation $*$ is associative.
	a * (b * c) &= a * (b+c-2) \\
		    &= a+(b+c-2)-2 \\
		    &= a+b+c-4 \\
	\end{aligned}
	$$
	Since $(a * b) * c = a * (b * c)$, the operation $*$ is associative.
\end{proof}


\end{problem}


\begin{problem}{2}
	Consider the set $G=\{x \in \mathbb{Q}: x \neq 1\}$. Define an operation $*$ on $G$ by

$$
a * b=a+b-a b \quad(a, b \in G) .
$$

(a) Show that $G$ is closed under $*$, that is, that for all $a, b \in G, a * b \in G$.
\begin{proof}
	Let $a,b \in G$. It is obvious that $a+b-ab \in \mathbb{Q}$. It remains to show that $a+b-ab \neq 1$. Suppose for the sake of contradiction that $a+b-ab = 1$.
	Then,
	$$
	\begin{aligned}
		a+b-ab &= 1 \\
		a-1+b-ab &= 0 \\
		(a-1)+(b-ab) &= 0 \\
		(a-1)(1-b) &= 0,
	\end{aligned}
	$$
	which implies that $a-1=0$ or $1-b=0$. If $a-1=0$, then $a=1$, which is a contradiction since $a \in G$. If $1-b=0$, then $b=1$, which is a contradiction since $b \in G$.
	Therefore, $G$ is closed under $*$.
\end{proof}

(b) Prove that $\langle G, *\rangle$ is a group.
\begin{proof}
	Let $a,b,c \in G$.
	From part (a) we know that $G$ is closed under $*$. It remains to show that $*$ is
	associative and that $G$ has an identity element and every element of 
	$G$ has an inverse. To show associativity, we must show that $(a * b) * c = a * (b * c)$. 
	This is true since:
	$$
	\begin{aligned}
		(a * b) * c &= (a+b-ab) * c \\
		&= (a+b-ab)+c-(a+b-ab)c \\
		&= a+b-ab+c-ac-bc+abc \\
		&= a+b+c-ab-ac-bc+abc \\
		&= a+(b+c-bc)-a(b+c-bc) \\
		&= a * (b+c-bc) \\
		&= a * (b * c).
	\end{aligned}
	$$

	Consider $0\in{G}$. Then, $a * 0 = a+0-a0 = a$. Also, $0 * a = 0+a-0a = a$. Therefore, $0$ is the identity element of $G$.
	% Consider $\frac{a}{a-1}$. Then, 
	To show that every element of $G$ has an inverse, we must show that for every $a \in G$,
	there exists $a' \in G$ such that $a * a' = 0$ and $a' * a = 0$.
	To determine $a'$, we must solve the equation $a+a'-aa'=0$ for $a'$. Thus, 
	$$
	\begin{aligned}
		a+a'-aa' &= 0 \\
		a'-aa' &= -a \\
		a'(1-a) &= -a \\
		a' &= \frac{-a}{1-a} \\
		a' &= \frac{a}{a-1}.
	\end{aligned}
	$$
	Then firstly, 

	$$
	\begin{aligned}
		% a * (a / (a - 1)) &= a + a / (a - 1) - a(a / (a - 1)) \\
		% &= a + a / (a - 1) - a^2 / (a - 1) \\
		% &= (a^2 -a +a-a^2) / (a - 1) \\)
		% &= 0 / (a - 1) \\
		% &= 0.
	%use ff instead of /
		a * (\frac{a}{a-1}) &= a + \frac{a}{a-1} - a(\frac{a}{a-1}) \\
		&= a + \frac{a}{a-1} - \frac{a^2}{a-1} \\
		&= \frac{a^2 -a(a-1) +a(a-1) -a^2}{a-1} \\
		&= \frac{0}{a-1} \\
		&= 0.
	\end{aligned}
	$$

	Also secondly,
	$$
	\begin{aligned}
		% (a / (a - 1)) * a &= a / (a - 1) + a - a(a / (a - 1)) \\
		% &= a / (a - 1) + a - a^2 / (a - 1) \\
		% &= (a - a^2 + a(a - 1) - a^2) / (a - 1) \\
		% &= 0 / (a - 1) \\
		% &= 0.
		a * (\frac{a}{a-1}) &= \frac{a}{a-1} + a - a(\frac{a}{a-1}) \\
		&= \frac{a}{a-1} + a - \frac{a^2}{a-1} \\
		&= \frac{a - a^2 + a(a - 1) - a^2}{a - 1} \\
		&= \frac{0}{a - 1} \\
		&= 0.
	\end{aligned}
	$$
	Therefore, $\frac{a}{a-1}$ is the inverse of $a$ in $G$. Thus, $\langle G, *\rangle$ is a group.

\end{proof}
\end{problem}

\begin{problem}{3}
	Let $G$ be any group. Prove that for all $a, b \in G$, if $a b=e$, then $b a=e$.
\begin{proof}
Let $a, b \in G$. Suppose that $a b=e$.
Then,
$$
\begin{aligned}
	a b &= e \\
	a^{-1} a b &= a^{-1} e \\
	e b &= a^{-1} e \\
	b b^{-1} &= a^{-1} b^{-1} \\
	e &= a^{-1} b^{-1} \\
	b a e &= b a a^{-1} b^{-1} \\
	b a &= b e b^{-1} \\
	b a &= b b^{-1} \\
	b a &= e.
\end{aligned}
$$
Therefore, if $a b=e$, then $b a=e$.
\end{proof}

\end{problem}


\end{document}
